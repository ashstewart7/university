% !TeX root = main.tex
\lecture{13}{Thu 13 Nov 2025 09:00}{Combining Probabilities}

Today we will arrive at:
\begin{itemize}
    \item The formula for $P(A \cap B)$ (Probability of A and B).
    \item Summing mutually exclusive events.
\end{itemize}

\section{More Set Theory}
We have a sample space $\Omega$, and a subset labelled $A$. We then have the remainder of $\Omega$ (the portion of $\Omega$ which is not in $A$), denoted "A Complement" - $A^C$ or $\bar{A}$. We also have a subset labelled $B$. 

We can define A using set builder notation, to slightly redundantly say "A is the set of all x'es which are in A":
\[
    A = \left\{x \mid x \in A\right\}
\]

There is some overlap between A and B. We denote this intersection as $A \cap B$.
\[
    A \cap B = \left\{x \mid x \in A \text{ and } x \in B\right\}
\]

Everything written in A or B (including the intersection) is called the union, $A \cup B$:
\[
    A \cup B = \left\{x \mid x \in A \text{ or } x \in B\right\}
\]
Note that ``or'' in standard language excludes both, i.e. you may have x or you may have y. In mathematics, we refer to this as XOR (exclusive or). ``Or'' by itself does allow for this case of both, so an item in A or B may be in A alone, B alone, or both (i.e. in the intersection).

We also have the empty set $\emptyset = \{\}$. If two sets have no common elements, the intersection is this empty set. We say that the events are mutually exclusive (they cannot both happen) and the sets are pairwise disjoint. The empty set is the complement of $\Omega$, $\emptyset = \Omega^C$.

\section{De Morgan's Laws}
De Morgan's Laws give us these relations:
\begin{equation}
    (A \cup B)^C = A^C \cap B^C
\end{equation}

\begin{equation}
    (A \cap B)^C = A^C \cup B^C
\end{equation}

This can be illustrated visually as follows:
\begin{figure}[H]
    \centering
    \includegraphics[width=0.75\textwidth]{figures/lec13-02.png}
     \caption{}
\end{figure}

\begin{figure}[H]
    \centering
    \includegraphics[width=0.75\textwidth]{figures/lec13-03.png}
     \caption{}
\end{figure}

\section{The Inclusion-Exclusion Principle}
The number of elements in A and B is given by:
\[
    |A \cup B| = |A| + |B| - |A \cap B|
\]
The last term is required to account for the intersection of A and B being included in A, and included in B. Therefore it double-counts the intersection, and we subtract it away.

The same is true of probability functions:
\[
    P(A) + P(B) = P(A \cup B) + P(A \cap B)
\]
In other words, the probability of A + the probability of B is the probability of A or B + the probability of A + B.



