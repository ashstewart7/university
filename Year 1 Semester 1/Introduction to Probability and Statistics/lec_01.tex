% !TeX root = main.tex
\graphicspath{{figures/}}
\lecture{1}{Wed 01 Oct 2025 12:00}{Introdution and Descriptive Statistics}

\subsection*{Course Welcome}
\begin{itemize}
    \item First half of the semester: Statistics
    \item Second hald the of semester: Probability
    \item All slides and notes on Canvas.
\end{itemize}

\textbf{Why Descriptive Statistics?}
If we want to share an interesting bit of data, sharing the whole data is going to be confusing. Instead, we can share a small number of stats which describe and summarise the data.

\subsection*{Sample Statistics}
One of the most simple is the number of samples ($N$), and the sample mean:

\[
    \text{Sample Mean: }\bar{x} = \frac{1}{N} \sum^N_{i=1} x_i
\]

We can also calculate the sample standard deviation as the average of mean squared error across the points in the sample:

\[
    \text{Sample STDev: } s_n^2 = \frac{1}{N} \sum_{i=1}^{N} (x_i - \bar{x})^2
\]

We can also use median or mode as measures of central tendancy. The mode is a poor estimator however (as it massively depends on how binning is done, for a continuous measurement), while the median is more resistant to outliers.