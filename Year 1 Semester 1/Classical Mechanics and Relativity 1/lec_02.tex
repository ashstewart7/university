% !TeX root = main.tex
\lecture{2}{Thu 02 Oct 2025 15:00}{Dimensional Analysis (contd.) and Vectors}
\subsection*{Continuation of Dimensional Analysis}
What if, in theory, we could build a system of units entirely from $c$, the speed of light, $G$, Newton's constant and $h$, the Plank Constant?

Cont. from Lec01, we can try to use this to work out the earliest possible cosmic time.

\[
    h = 6.6 \times 10^{-34}Js
\]
\[
    G = 6.67 \times 10^{-11} N m^2 / kg^2
\]
\[
    c = 3 \times 10^8 m/s
\]

Dimensionally:
\[
    [h] = \frac{ML^2}{T}
\]
\[
    [G] = \frac{L^3}{T^2 M}
\]
\[
    [c] = \frac{L}{T}
\]

We want to use these to build out a time unit, so:
\[
    [h^u G^v c^z] = T
\]
\[
    \left(\frac{ML^2}{T}\right)^u \left(\frac{L^3}{T^2 M}\right)^v \left(\frac{L}{T}\right)^z = T
\]

\[
    M^{u-v} L^{2u + 3v + z} T^{-u - 2v -z} = T
\]

Solving for:
\[
    u-v = 0
\]

\[
    2u+3v + z = 0
\]

\[
    -u - 2v - z = 1
\]

Gives us:

\begin{align}
    u &= \frac{1}{2}\\
    v &= \frac{1}{2}\\
    z &= \frac{-5}{2}
\end{align}

$t_p = \sqrt{\frac{Gh}{c^5}}$ and plugging in the values for G, h, c gives us a value of time, which the earliest possible cosmic time equal to about $10^{-43}s$


\subsection*{Plank Energy}
Doing the same process for energy gives us (this time, the plank energy is the energy at which traditional theories of physics break down):
\[
    E_p = \frac{hc^5}{G}^{0.5} \approx 10^9 J
\]

On the other hand, the LHC manages about 10TeV, which is orders of magnitude smaller than this, so the LHC cannot accurate simulate energies of this magnitude.

\subsection*{More Vectors}
Again, vector notation will be $\vec{a}$. We define the x, y, z unit vectors as $\underline{\hat{e_x}}, \underline{\hat{e_y}}, \underline{\hat{e_z}}$.

We can therefore define any vector as:
\[
    \vec{a} = a_x \underline{\hat{e_x}} + a_y \underline{\hat{e_y}} + a_z\underline{\hat{e_z}}.
\]

The length of a vector is again $|\vec{a}|$.

\subsection*{Vector Multiplication}
Given $\vec{a}$ and $\vec{b}$ we can define the dot (scalar) product and the cross (vector) product

\[
    \vec{a} \cdot \vec{b} = |a||b| \cos \theta
\]

Say we want to know the component of a vector along an axis, we can do the following (eg for x):
\[
    \vec{a} \cdot \underline{\hat{e_x}} = a_x
\]

For the vector product, we can define:
\[
    \vec{a} \times \vec{b} = |a||b| \sin(\theta) \hat{j}
\]
As the vector perpendicular to the plane containing a and b. It is in the direction such that\footnote{TODO, fix}. Theta is the angle between a and b, while j is the unit vector in the direction the new vector will point.

\subsection*{Solar Energy Example}
The world yearly energy usage is about $180,000 \text{TWh}$, which is about $5 \times 10^{20}J$ total. Is it (theoretically) possible to get this all from solar energy? We can check using an approximate order of magnitude calculation.

The Sun's total luminosity is $L_{\odot} = 3.8 \times 10^{26}$. This energy is radiated in a spherically symmetric way (we assume). Therefore the energy per time, per unit surface is (using 1AU for distance):
\[
    \frac{L_{\odot}}{4\pi \times (1.5 \times 10^6)^2}
\]

Which is approximately (using order of magnitude):
\[
    \frac{3.8 \times 10^{26}W}{10 \times 10^{22}m^2} \approx \frac{1kW}{m^2}
\]
This is true in ideal conditions, and real energy supply is lower (due to clouds, atmosphere etc). 

If we totally covered the earth's surface area ($A_{\text{surface}} \approx \pi R_{\odot}^2$) which is approximately:
\[
    A_{\text{surface}} \approx \pi \times (6\times10^3\times10^3m)^2 \approx 10^{14}m^2
\]

Therefore total energy recieved is approximately:
\[
    P = \frac{1kW}{m^2} \times 10^{14}m^2 \approx 10^{17}W
\]

And to power the world:
\[
    E = \frac{5 \times 10^{20}J}{3 \times 10^7s} \approx 10^{13}W
\]

So, it's theoretically possible, if we could cover enough of the world in solar panels and if we could perfectly capture the sun's energy without losing some to sources such as clouds, atmosphere, areas of the ocean we cannot cover in solar panels etc.

