% !TeX root = main.tex
\graphicspath{{figures/}}
\lecture{3}{Thu 04 Sep 2025 12:00}{Kinematics Introduction}
For kinematics, we'll treat all objects as points and disregard aspects like rotation/the physical sie of the body etc.

Given some point, we can define its position as a fucntion of time $\vec{r}(t)$, and velocity as the derivative wrt time of this:
\[
    \vec{v}(t) = \frac{d \vec{r}}{dt}
\]

And acceleration:
\[
    \vec{a}(t) = \frac{d \vec{v}}{dt} =\frac{d^2 \vec{r}}{dt^2}
\]

\subsection*{Position from Unit Vectors}
We can define:
\[
    \vec{r}(t) = r_x(t) \hat{e_x} + r_y(t) \hat{e_y} + r_z(t) \hat{e_z}\\
    = \sum_{j=1}^{3} r_j(t) \hat{e}_j
\]

So:
\[
    \frac{d \vec{r}}{dt} = \frac{d}{dt} \left(\sum_{j=1}^{3} r_j \hat{e}_j\right)
\]
\[
    = \sum_j \frac{d}{dt}(r_j \hat{e}_J)
\]

Note: Taking the derivative of a vector wrt time is looking at how the variable changes in some infinitestimal time. This can be a change in direction, and/or a change in magnitude. To differentiate a vector we can differentiate it component-wise.

\subsection*{Cartestian and Polar}
Instead of representing a point as x and y components (in 2D), we can instead define it as a distance from the origin $r$ and the angle this distance line forms with the positive x-axis $\theta$.

Therefore (by basic right angle trig) $x = rcos \theta$, $y = rsin \theta$, and hence:

\[
    \vec{r} = r \cos \theta \hat{e}_x + r \sin \theta \hat{e}_y
\]

So:
\[
    \vec{u(t)} = \frac{d \vec{r}}{dt} = \frac{d}{dt} (r \cos \theta) \hat{e}_x + \frac{d}{dt} (r \sin \theta) \hat{e}_y
\]

\[
    = \left(\dot{r}\cos \theta + r (-\sin \theta) \dot{\theta}\right) \hat{e}_x + \left(\dot{r} \sin \theta + r (\cos \theta) \dot{\theta}\right) \hat{e}_y
\]
\[
    = r \left(\cos \theta \hat{e}_x + \sin \theta \hat{e}_y\right) + r \dot{\theta} \left(-\sin \theta \hat{e}_x + \cos \theta \hat{e}_y\right)
\]


\subsection*{Example}
Lets model a particle, in a single dimension moving with constant acceleration ($a_0$) along a line. What is x(t)?

\subsection*{Going Backwards}
Lets say we have some body with $a(t) = k t^3$. What is the position function $x(t)$?

\[
    a = \frac{dv}{dt} = kt^3
\]

\[
    v = \int_{a}^{b} kt^3 \, dx
\]

