% !TeX root = main.tex
\lecture{4}{Fri 24 Oct 2025 12:00}{Atomic Energy Levels and Atomic Spectra}

In this lecture:
\begin{itemize}
    \item The spectra of light emitted and absorbed by electrons in:
    \begin{itemize}
        \item Hydrogen (simple)
        \item Larger atoms (not simple\ldots)
    \end{itemize}
    \item Electronic shells and orbits.
\end{itemize}

\section*{Spectra}
Electrons in an atom can hold discrete values ``levels'' of energy. As electrons go up or down these levels they must absorb or will emit a photon. This emission causes a discrete spectra of emitted frequencies, unique to the element causing it.
\begin{figure}[H]
    \centering
    \includegraphics{figures/lec04-01.png}
     \caption{Example emission spectra.}
\end{figure}

Each of the transitions from one energy level to another have a discrete change in energy (measured in eV), therefore each transition will have a discrete wavelength of produced photons.

From the Bohr model, he postulated (from experimental observations) that light can \emph{only} be absorbed or emitted when an electron goes up or down a discrete energy level (exitation or relaxation). At the lowest energy level (ground state).

It is important to note that the transitions in energy levels are not spatial - changing the energy level of an electron does not physically change its position (despite what the diagrams may imply).

\section*{Absorption and Emissions}
\begin{figure}[H]
    \centering
    \includegraphics{figures/lec04-02.png}
     \caption{}
\end{figure}

An electron is excited up an energy level, leaving a vacancy behind. Some time after, the electron drops back down into the ground state, emitting the energy in terms of a photon. By conservation of energy:
\[
    E_{\text{in}} = E_\text{out} = E_2 - E_1 = hf = \frac{hc}{\lambda}
\]



