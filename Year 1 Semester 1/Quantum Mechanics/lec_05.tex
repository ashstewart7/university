% !TeX root = main.tex
\lecture{5}{Fri 31 Oct 2025 12:00}{X-Ray Production and Diffraction}
In this lecture:
\begin{itemize}
    \item Production of x-rays
    \item How to measure their wavelength
    \item Bragg scattering of x-rays by crystals and Bragg conditions.
\end{itemize}

\section*{Production of X-Rays}
From the blackbody spectra lecture, we saw that something becoming hotter affects the wavelength of emission. Surely we, therefore, can just heat something hot enough for the peak of the emission spectra to be in the x-ray range?

Unfortunately not, this works, but would need the black body to be 1MK.

The more practical alternative to to fire high energy electrons (keV range) into a metal target in a vacuum.

\section*{What are they?}
Electromagnetic radiation (light) outside of the visible spectrum. It sits beyond the UV.

Approximately 0.1keV ($\lambda = 10nm$) to 100keV ($\lambda = 0.01nm$), these are called soft and hard x-rays respectively.

\section*{Measuring Wavelength}
How can we measure or select x-ray wavelengths? With visible light, we can use a prism to turn wavelength into an angle and measure the angle. However, a prism works for visible light because of its high refractive index. Unfortunately, for x-rays, $n \approx 1$ for all materials.

We therefore need some other way to split up wavelengths into angles, for which we can use a diffraction grating. Different wavelengths will be diffracted at different angles.

Another problem arises however\dots We need the slit separation $d$ to be $d \approx \lambda$ for a diffration grating to work. For x-rays, $\lambda \approx 10^{-10}m$, which is approximately the width of an atom. Good luck making that grating\dots

Luckily, nature has made these gratings for us --- crystalline materials! Strontium titanate ($SrTiO_3$) has an approx 0.389nm spacing between strontium atoms. We can therefore use this crystal lattice as a diffraction grating.

\section*{X-Ray Diffraction from Crystals}
We treat the x-rays as EM waves (and ignore weird photon stuff for now) and we assume that each atom scatters independantly. We also assume that the atom absorbs the x-ray and later reemits it, in all directions uniformly (at photon level, the emission is in a random direction, therefore we probabilistically treat it as in all).

We want to find the angles where \emph{constructive interference} occurs. Zero intensity everywhere else due to very very high number of slots.

The Bragg Conditions determine where this interference happens. Consider a single plane of equally spaced atoms.
