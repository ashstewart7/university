% !TeX root = main.tex
\lecture{8}{Fri 21 Nov 2025 12:00}{Wave-Particle Duality}
In this lecture:
\begin{itemize}
    \item Particles acting as waves and vice-versa - depends on how we look at/measure them.
    \item Infinite potential wells - quantisation of energy in bound states, zero-point energy.
    \item The Heisenberg Uncertainty Principle.
\end{itemize}


\section{Single Slit (Fraunhofer) Diffraction}
We have seen particles behaving like waves (i.e. electron diffraction) and waves behaving like particles (i.e. the photoelectric effect). This seems to depend on how we look at them - different experiments give different results. 

Broadly, they seem to be wave-y when in motion (i.e. interference), and particulate when detected. In a single-slit experiment, we can see both at the same time. We fire electrons through a single slit, and measure the diffraction pattern with a screen or movable detector:
\begin{figure}[H]
    \centering
    \includegraphics[width=\textwidth]{figures/lec08-01.png}
     \caption{}
\end{figure}

We can do this experiment with either photons, or with electrons and we observe the same result: both yield the same interference pattern due to superposition of waves cancelling each other out. 

What if we turned the source way down and fired a single photon/electron at a time. We would therefore detect individual particles on our detector as they arrive one-by-one. Surely, therefore, we wouldn't get a diffraction pattern? We do, however, get a pattern\ldots What are they interfering then, if there's no other particles to interfere with? It turns out the particle interferes with itself.

We view the intensity profile as a probability distribution of any one particle impacting here. The areas of the profile with a higher probability have a higher measured intensity, to be expanded on next lecture. The amplitude of a particle wave at a point represents the probability of the particle being found at that point.

\section{Infinite Potential Wells}
Given a particle in an infinite potential well, how is this impacted by wave-like behaviour of the particle?

\begin{figure}[H]
    \centering
    \includegraphics{figures/lec08-02.png}
     \caption{}
\end{figure}

Classically, we can think of this as a squash court with a particle bouncing back and forth between the two left and right boundaries. The particle is stuck between zero and L, with potential energy measured on the y-axis. Note that that since the y-axis is energy - it is not spatial. The only possible spatial movement of the particle is left-to-right.

Quantum mechanically, we have the following rules:
\begin{itemize}
    \item The particle has finite energy. It therefore only has amplitude (probability of finding the particle) at this one spot.
    \item The wave function of a particle has to be continuous throughout. Discontinuities break things.
\end{itemize}

As a result of these two rules, the wavefunction $\psi(x)$ must be zero at the boundary points $0$ and $L$. We also assume that we can define any wavefunction as a sine wave:
\[
    \psi(x) = A \sin \frac{n \pi x}{L}, \quad n = 1, 2, 3, \ldots
\]
Therefore:
\[
    \lambda_n = \frac{2L}{n}
\]
We can generate more complicated shapes using Fourier Decomposition, where we define a more complex wave function as the sum of (potentially infinitely) many sine waves.

Considering this wave function at multiple different energies (infinitely many as we're in an infinite well):
\begin{figure}[H]
    \centering
    \includegraphics{figures/lec08-03.png}
     \caption{Note: Bigger $n$ is higher energy.}
\end{figure}

We know from last lecture that we can relate wavelength and momentum, so:
\[
    p = \frac{h}{\lambda} \implies p_n = \frac{n h}{2L}
\]
\[
    E = \frac{p^2}{2m} = \frac{n^2 h^2}{8mL^2} 
\]

Note that $E \propto n$ in an infinitely deep 1D potential well. Compare this to $E \propto \frac{1}{n^2}$ in a H atom.

Energy is uniquely determined for each quantum state $\psi_n(x)$ for the particle in the well. If we consider momentum: classically we say the particle is moving either to the left \emph{or} or the right, with a known magnitude but an unknown direction (if we take a single observation as a snapshot in time).

For n=1:
\[
    p_x = \pm \frac{h}{2L}
\]

Which is effectively zero momentum but with an uncertainty of $\Delta p_x = \frac{h}{2L}$. The expectation value (average value):
\[
    \langle p_x\rangle = 0
\]

And for $x$ (somewhere in the well):
\[
    \langle x\rangle = \frac{1}{2}L
\]
With an uncertainty of:
\[
    \Delta x = \pm \frac{1}{2}L
\]

\section{Shrinking the Well}
What happens if we shrink the well? Both momentum and energy are inversely proportional to cavity length:
\[
    p \propto \frac{1}{L}
\]
\[
    E \propto \frac{1}{L^2}
\]

Therefore they both go up. This is weird in classical mechanics. We now know less about momentum (higher uncertainty) but more about the particle's location (smaller set of possible values, lower uncertainty than before). This isn't a proof, but a hint towards the Heisenberg Uncertainty Principle.

\section{Heisenberg Uncertainty Principle}
\begin{equation}
    \Delta x \Delta p_x \geq \frac{\hbar}{2}
\end{equation}
 
Where: $\hbar = \frac{h}{2 \pi}$. In QM, there is a limit to how well we can know two conjugate (effectively paired, formal word to mean a pair of variables where knowing one of them perfectly precludes knowing the other) observables simultaneously. This also applies to energy w.r.t. time:
\[
    \Delta E \Delta t \geq \frac{\hbar}{2}  
\]

Note that in a 2D system, $\Delta p_x \Delta y$ is unrestricted as they are independent and non-conjugate principles. In our example:
\[
    \Delta x \Delta p_x = \frac{L}{2} \frac{h}{2L} = \frac{h}{4} \geq \frac{h}{4 \pi}
\]

The single slit diffraction is a good example of this. If we measure the $x$ of a photon by passing it through a slit, we lose information on its momentum and it spreads out along $x$. 

\begin{figure}[H]
    \centering
    \includegraphics{figures/lec08-04.png}
     \caption{}
\end{figure}


\section{Conclusions}
\begin{itemize}
    \item Wave-particle duality allows particles to both diffract as if waves and to be quantised like particles.
    \item Wave-like particles explains why the energy in a bound system has to be quantised (as shown before) due to standing waves in a potential well.
    \item Energy levels depend on the shape of the potential well.
    \begin{itemize}
        \item $E \propto n^2$ in an infinitely deep 1D well.
        \item $\frac{1}{n^2}$ in a H atom.
    \end{itemize}
    \item We cannot know a particle momentum and position simultaneously to perfect precision, we are limited that if we know one very very accurately, we cannot know the other very well.
\end{itemize}















