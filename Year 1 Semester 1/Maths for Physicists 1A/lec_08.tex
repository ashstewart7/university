% !TeX root = main.tex
\lecture{8}{Mon 13 Oct 2025 12:00}{More Planes}

\section*{Recap}
Given the origin $O$, a point on the plane $O'$ and a vector $\vec{a}$ between them, we can take two vectors $\vec{b}$ and $\vec{c}$ from this point (which are not parallel). Using some combination of these two vectors, we can reach any point on the plane:

\[
    \vec{r}(s, t) = \vec{a} + s \vec{b} + t \vec{c} 
\]

This is the parametric equation of a plane, and is very robust. We can describe a flat plane in any dimensional space using this.

We can also define the scalar equation of a plane. Given these same two vectors, we can define a normal vector $\vec{\hat{n}}$ which is perpendicular to any vector that sits within the plane. We can construct this by using the cross product:
\[
    \vec{n} = \vec{b} \times \vec{c}
\]

Given some generic point $P$:
\[
    \vec{OP} = \vec{a} + \vec{O'P}
\]

And:
\[
    \underline{r}(s, t) = \underline{a} + s \underline{b} + t \underline{c}
\]

We have:
\[
    (\underline{b} \times \underline{c}) \cdot \underline{r} = (\underline{b} \times \underline{c}) \cdot \underline{a} + s(\underline{b} \times \underline{c})\cdot \underline{b} + t(\underline{b} \times \underline{c}) \cdot \underline{c}
\]

Which (as a vector dotted with itself is 0) simplifies to (using $\underline{b} \times \underline{c} = \underline{n}
$):
\[
    \underline{n} \cdot (\underline{r} - \underline{a}) = 0
\]
