\lecture{2}{Thu 02 Oct 2025 13:00}{Wave Functions}

\subsection*{Sine Waves}
\paragraph{Mechanical Waves} A mechanical wave is a disturbance through a medium. It's formed of a single wave pulse or a periodic wave. 

Mechanical Waves have the following properties:
\begin{enumerate}
    \item Transverse: Where displacement of the medium is perpendicular to the direction of propogation.
    \item Longitudinal, displaccement of hte medium is in the same direction as propogation.
    \item Propogation depends on the medium the wave moves through (i.e. density, rigidity)
    \item The medium does not travel with the wave.
    \item Waves have a magnitude and a direction.
    \item The disturbance travels with a known exact speed.
    \item Waves transport energy but not matter throughout the medium.
\end{enumerate}


\subsection*{Wave Functions}

We want to define a wave function in terms of two variables, $x$ and $t$. In any given moment, if we consider a single point on the wave (i.e. $t = 0$), and wait a short while, the wave will have travelled to some $t = t_1 > 0$.

In order to quantify displacement, we therefore want to specify both the time, and the displacement. This will let us find the wave speed, acceleration and the (new) wave number.

We are also able to talk about the velocity and acceleration of individual particles on the wave.

\subsection*{Wave Function for a Sine Wave}
Consider a sine wave. We want to find a wave funtion in the form $y(x, t)$. Consider the particle at $x = 0$.

We can express the wave function at this point as $y(x = 0, t) = A \cos \omega t$. However we want to expand this to any general point. Now consider a point (2) which is one wavelength away. We know the behaviour of particle 1 is mirrored by particle 2 (with a time lag).

Since the string is initially at rest, it takes on period ($T$) for the propogation of the wave to reach point 2, therefore point 2 is lagging behind the motion of point 1 however. The wave equation is therefore (if particle two has $x = \lambda$) $y(x = \lambda, t) = A \cos (\omega t - \frac{\pi}{2})$. 

For arbitrary $x$, $y(x, t) = A \cos(\omega t - \frac{x}{\lambda}\cdot 2 \pi)$ to account for this delay. This quantity is called the wave number:

\[
    \text{Wave Number: } k = \frac{2 \pi}{\lambda}
\]

So:
\[
    y(x,t) = A \cos(\omega t - k x)
\]
\[
    = A \cos(kx - \omega t)
\]

Note the second step is possible as cos is an even function. If $k > 0$, the wave travels in the positive x. If $k < 0$, the wave travels in the negative x direction. Again, $\omega = 2 \pi f$

\subsection*{Displacement Stuff}
Considering a point (starting at equi), the time taken for the particle on the sin wave to reach maximum displacement, minimum displacement and back takes the time period $T$. The speed of the wave is distance travelled over the time taken. We take the distance to be the wavelength $\lambda$, as we know the time by definition this takes is one time period $T$. Therefore wave speed $v$ is:
\[
    v = \frac{\lambda}{T} = \lambda f
\]

Since $\lambda = \frac{2 \pi}{k}$ and $f = \frac{\omega}{2 \pi}$ (as $\omega$ is defined as $\frac{2 \pi}{T}$), we can also write:

\[
    v = \frac{2 \pi}{k} \cdot \frac{\omega}{2 \pi} = \frac{\omega}{k}
\]

\subsection*{Particle Velocity}
We can also determine the velocity of individual particles in the medium. We can use this to determine the acceleration.

We know that
\[
    y(x, t) = A \cos(kx - \omega t)
\]

The vertical velocity $v_y$ is therefore given by:
\[
    v_y = \frac{dy(x, t)}{dt}
\]

Which is unhelpful (as we can't differentiate two variables at once), we can slightly cheat this by looking at purely a certain value of x, and therefore treating x as constant (to get a single variable derivative).
\[
    v_y = \frac{dy(x, t)}{dt} \Bigr\rvert_{x = \text{const.}}
\]


However this is notationally yucky, so we therefore use the notation:
\[
    \frac{\partial y(x, t)}{\partial t}
\]

To represent the same thing. Finally (carrying out the partial derivative):

\[
  v_y(x, t) = \frac{\partial y(x, t)}{\partial t} = \omega A \sin(kx - \omega t)
\]

\subsection*{Particle Acceleration}
We can work out particle acceleration (transverse acceleration) by differentiating in the same manner again:
\[
    a_y(x, t) = \frac{\partial^2 y(x, t)}{\partial t^2} = \frac{\partial}{\partial t} \left(\frac{\partial y(x, t)}{\partial t} \right) = - \omega^2 A \cos(kx - \omega t) = \omega^2 y(x, t).
\]

