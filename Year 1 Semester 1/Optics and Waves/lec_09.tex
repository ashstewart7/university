% !TeX root = main.tex
\lecture{9}{Thu 30 Oct 2025 13:00}{Optics Part Two}

\section*{Key Principles}
There are two main principles that we'll use:
\begin{enumerate}
    \item \textbf{Hugen's Principle}
    \begin{itemize}
        \item Each point on a wavefront serves as a source of spherical secondary \emph{wavelets} that advance with speed and frequency identical to the primary wave.
        \item If we consider each point on a wave as a spherical wavelet source, the wavefront is given by a line tangent to all of these.
    \end{itemize}
    \item \textbf{Fermat's Principle}
    \begin{itemize}
        \item The actual path taken by a beam of light is the one which takes the least time to traverse.
        \item i.e. $dt/dl = 0$
        \item Light could theoretically take an infinite number of possible routes between any two points, but in practice (in the same material) it will travel in a straight line as this is the fastest route.
        \item When light is travelling through an interface (i.e. an air to glass boundary), the fastest point is no longer a straight line, as refraction appears.
    \end{itemize}
\end{enumerate}

\section*{Reflection}
Lets consider some mirror as a reflecting surface, with light travelling from point A to point B. There are multiple different theoretical paths that the light could take, so Fermat's principle requires the shortest possible time taken. 

\begin{figure}[H]
    \centering
    \includegraphics[width=0.5\textwidth]{figures/lec10-01.png}
     \caption{}
\end{figure}

We require, by Fermat's Principle (as same material, so shortest time is simply shortest path):
\[
    AC + CB = \text{min}
\]


By considering congruent triangles (let O be the midpoint of B and B'), therefore:
\[
    AC + CB = AC + CB'
\]

Provided $ACB'$ is a straight line, $\theta_i = \theta_r (= \theta_{r'})$. Therefore the angles of incidence and reflection must be equal.

\section*{Refraction}
When light is travelling through a material, the speed of light in each is different. Therefore, a straight line is no longer the fastest path.

\begin{figure}[H]
    \centering
    \includegraphics{figures/lec10-02.png}
     \caption{}
\end{figure}

Where:
\begin{itemize}
    \item $n_i$ is the incident index.
    \item $n_t$ is the post-refraction index.
    \item $\theta_i$ is the angle of incidence.
    \item $\theta_t$ is the angle of refraction.
    \item Light is travelling from point A to point B.
    \item Distances are given per diagram.
\end{itemize}

The aim is to be able to find the true trajectory (per Fermat) that takes the shortest amount of time. By s = d/t, for given distance values:

\[
    t = \frac{\sqrt{h^2 + x^2}}{v_i} + \frac{\sqrt{b^2 + (a-x)^2}}{v_t}
\]

Where $v_i, v_t$ are speeds of light in the respective media. t is a function of x, with constants, so to find a min value we differentiate wrt x and solve for minima:
\[
    \frac{dt}{dx} = \frac{x}{v_i \sqrt{h^2 + x^2}} + \frac{-(a-x)}{v_t \sqrt{b^2 + (a-x)^2}} = 0
\]

Subsituting values for distances, we get:
\[
    \frac{\sin \theta_i}{v_i} = \frac{\sin \theta_t}{v_t}
\]

And since $n = c/v$, we get:

\begin{equation}
    n_i \sin \theta_i = n_t \sin \theta_t
\end{equation}

The shortest (wrt time) path must have angles which satisfies this. This is called Snell's Law.


\section*{Connecting Fermat and Huygen}
Speed of light in a medium is less than in a vacuum, this is categorised by the index of refraction, $n$:
\[
    n = \frac{c}{v}
\]
Where c is speed of light in a vacuum, and v is speed of light in the medium. i.e. for water, $n = 1.333$, for air, $n = 1.0003$.

Suppose we have a wavefront travelling as a series of wavelets. As soon as each wavelet hits the glass, they slow down. This leads to an uneven distribution of speeds across the wavelength, as some have slowed down and some have not. Therefore, the wave bends around and we get a new wavefront at a different angle to the previous wavefront.

Note the condition where the light travels perpendicular to the interface, i.e. $\theta_i = \theta_t = 0$. In this case, all wavelets hit the medium and change speed at exactly the same time, therefore there is no difference in velocity across the wavefront, so no direction change.

\section*{But how?}
How does light actually ``choose'' which path to take? Apparently it just does and we have to wait to find out\dots

\section*{Combining Reflection and Refraction}
When light strikes a boundary surface, there is two components - both reflected and transmitted (just like on a string). To determine this, we have to use the wave equation:
\[
    E = E_0 \cos(kx - \omega t)
\]

Noting that instead of traditional mechanical amplitude we use the amplitude of the electromagnetic field at this point, $E_0$. For now, we ignore polarisation as a potential scenario.

\section*{More Index}
We note that since $v = f \lambda$, the index of refraction will change depending on the frequency of light:

\begin{figure}[H]
    \centering
    \includegraphics[width=\textwidth]{figures/lec10-03}
     \caption{}
\end{figure}

When we shine light of multiple frequencies (i.e. white light), the higher frequency light bends further than the lower frequency light. This causes dispersion of light, i.e. the formation of a rainbow out of a prism. This arises because each frequency has a different speed and therefore a different change in $\theta$.

If we shine pulses of light of multiple frequencies down a fibre optic line, the higher frequencies will have a higher n, hence a lower speed and will arrive later.

This allows us to explain the formation of sunsets and rainbows etc. In the case of a sunset, when we see the sun just above the horizon, the sun has actually set just below the horizon. We cannot therefore see the sun via direct line of sight, and yet we can still see it as if we could?

This is because the light from the sun is refracted and bends towards us. The atmosphere is higher density at the bottom, and lower density as altitude increases. The amount of refraction depends on density, hence changing the index of refraction. We can say that n is a function of y, where y is height.

We imagine it as being actually present precisely where we see it, because our brain does not account for this refraction and extrapolates the light as a straight line. We therefore see the sun has just about to set, when in reality the sun physically has set below the horizon.












