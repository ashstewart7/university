% !TeX root = main.tex
\lecture{7}{Thu 04 Sep 2025 01:00}{Energy and Power of Waves}

\section*{Kinetic Energy for a Sine Wave}
Assuming a sine wave created by a harmonic oscillator.

\[
    KE = \frac{1}{2}mv^2
\]
\[
    (KE)_\text{max} = \frac{1}{2}m v_\text{max}^2
\]
\[
    (KE)_\text{max} = \frac{1}{2}(\mu dx)(\omega A)^2
\]

And then to get the total wave energy, we can integrate over lamda:
TODO



\section*{And Power}
\[
    \text{Power} = \frac{\text{Energy}}{\text{Time}}
\]
\[
    P = \frac{\frac{1}{2}(\omega A)^2 \mu \lambda}{T}
\]
\[
    = \frac{1}{2} (\omega A)^2 \mu v
\]

\section*{Example}
Considering a standing wave with $\lambda = 2L$:

\[
    y_t = y_1 + y_2
\]
\[
    \implies (KE)_t = (KE)_1 + (KE)_2
\]

Where $y_1$ and $y_2$ are identical except for their direction, but they carry the same kinetic energy.
The KE of a standing wave is the sum of the two waves that make it up.

Note: Normally, A is the amplitude of the travelling wave (hence 2A is the amplitude of the standing wave created by them), however it is sometimes ambigous what is being referred to by A.

\section*{Interference}
\textbf{Superposition}: When waves overlap in the same region, the resulting wave is the algebraic sum of waves (they interfere)

Consider two waves:

