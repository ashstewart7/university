% !TeX root = main.tex
\lecture{3}{Thu 17 Oct 2025 12:00}{Particle Nature of Light}

In this lecture:
\begin{itemize}
    \item The photoelectric effect.
    \item Compton scattering.
\end{itemize}
Which are two examples where classical theory (light as a wave) break down.

\section{The Photoelectric Effect}
When shining ultraviolet light on a metal surface, electrons are emitted. This is the photoelectric effect. 

Why are we not bombarded by electrons in daily life? For the electron to fly off, we must be in a vacuum. Otherwise, it'll immediately strike an air molecule and be absorbed.

\textbf{Photoelectric Effect Background}
\begin{itemize}
    \item Discovered by Hertz, 1887
    \item Thomson (1889) went further, so did Lenard (1902) and others.
    \item Einstein won his Nobel Prize for explaining this, not from relativity.
\end{itemize}

\begin{figure}[H]
    \centering
    \includegraphics{figures/lec03-01.png}
     \caption{A circuit diagram for measuring the photoelectric effect.}
\end{figure}

The above setup would be encased in a glass ball (containing a vacuum), with a setup like this:

\begin{figure}[H]
    \centering
    \includegraphics[width=0.75\textwidth]{figures/lec03-02.png}
     \caption{Experimental Setup}
\end{figure}

\subsection{Results}
\textbf{Result One - Changing Intensity}\\
For fixed UV wavelength, increasing the intensity of light increases the measured photocurrent:
\begin{figure}[H]
    \centering
    \includegraphics{figures/lec03-03.png}
    \caption{}
\end{figure}
\begin{itemize}
    \item Increasingly negative potential the cathode decreases photocurrent. At some potential $v_s$ applied to the circuit (the ``stopping potential'') this current drops to zero.
    \item Potential does not affect electron emission, however adding potential causes an electric field which effectively blows electrons back towards the anode. The stopping potential is when this electric field is perfectly strong to prevent electrons from reaching the cathode and causing a current.
    \item The fact this can happen consistently (i.e. no current means no electrons made it through) implies that there must be some maximum kinetic energy these electrons can have ($KE_{max} = eV_S$).
    \item The stopping potential is independent of UV intensity. More UV makes current increase, but does not change stopping potential (i.e. it does not give more energy to each electron, they each have the same energy). This does not make sense classically. Classically we would expect adding more energy to cause emitted electrons to have more energy, therefore changing the stopping potential.
\end{itemize}

\textbf{Result Two - Changing Wavelength}
\begin{figure}[H]
    \centering
    \includegraphics{figures/lec03-04.png}
    \caption{}
\end{figure}

We have to reach some baseline threshold frequency $f_0$ before we see any photocurrent. After this, increasing wavelength increases photocurrent (and hence KE of emitted electrons) linearly.

\begin{itemize}
    \item For a given metal, we find the threshold frequency $f_0$, below which there is no emission of electrons (no current). If below the frequency $f_0$, intensity is irrelevant. This contradicts classical mechanics which would suggest that turning up the light intensity would supply more (and potentially sufficient) energy.
    \item Above the threshold, the energy of individual emitted photons depends on UV frequency and not intensity (by result one).
\end{itemize}

\subsection{Conclusions}

\textbf{Classically}

Classically, we expect energy to be proportional to intensity, $\therefore$ $v_s$ should increase with greater intensity. We also expect there to be no link between between frequency and energy, hence no threshold frequency.Instead of a threshold frequency, we'd expect a time delay as electrons ``soak up'' energy to reach the required threshold.

In theory, great, in practice \emph{this is not observed.}

\textbf{Einstein's Proposal}

Energy in light comes from photons with energy $E = hf$. There is a minimum energy required for an electron to be able to escape from the metal. This minimum energy is called the work function $\phi$.
\[
    KE_\text{max} = hf - \phi = eV_s
\]

Now:
\begin{itemize}
    \item Higher intensity means more of the same particles (more photons), but the energy of each is unchanged.
    \item $E = hf$ so frequency changes energy (as observed).
    \item The Bohr model says that an electron can only have certain electron energy transitions when the correct energy is supplied (an electron cannot gradually soak up energy). This explains why there is a cutoff below the work function, and no observed time delay (as the ``soaking up'' that causes the delay does not happen). Either an incoming photon has sufficient energy, or it does not. Having more photons does not help.
    \item The first incoming photons immediately releases an electron (assuming the incoming light has sufficient energy), therefore there's no time delay.
\end{itemize}

\subsection{In Practice}
\begin{figure}[H]
    \centering
    \includegraphics{figures/lec03-05.png}
     \caption{Sodium photocurrent measurements by Robert Millikan}
\end{figure}


\section{Compton Scattering}
Compton Scattering is the scattering of photons off atoms.

\begin{figure}[H]
    \centering
    \includegraphics{figures/lec03-06.png}
     \caption{A Compton Scattering experimental setup.}
\end{figure}

The surprising result is that two wavelengths were observed (not just the original) - $\lambda_1, \lambda_2$, where $\lambda_1$ is the original and $\lambda_2$ is different. Classically this is hard to explain and $\lambda$ should not change.

The difference between these two wavelengths increases with scattering angle $\theta$. This can be explained if the x-ray beam is a stream of photons, but not classically.

Possible options when a photon collides:
\begin{itemize}
    \item Scattering off the whole atom (Rayleigh Scattering). The photon hits a tightly bound electron that cannot move independently. The photon effectively scatters off the entire atom. Since $M_\text{atom} \gg m_e$, the recoil energy of the atom is minimal and the photon loses almost no energy, hence the wavelength is effectively unchanged. This gives us the existing $\lambda_1$ peak.
    \item Scattering off a free electron (Compton Scattering.): The photon hits a loosely bound electron which has a much lower binding energy, so the electron is knocked loose. The photon transfers momentum and energy to this electron, leading to a loss of energy and therefore change of wavelength. This gives the new $\lambda_2$ peak.
\end{itemize}

\begin{figure}[H]
    \centering
    \includegraphics{figures/lec03-07.png}
     \caption{The observed results}
\end{figure}

\section{Deriving Compton's Equation}
Given an incoming photon with with energy $E_1$, wavelength $\lambda_1$ and momentum $\underline{p}_1$. This strikes an electron and is deflected by angle $\theta$. The electron is deflected by some angle $\phi$ such that momentum is conserved. The new deflected photon has $E_2$, $\lambda_2$, $\underline{p}_2$.

We must consider relativistic effects here given the high speed ($E^2 = p^2c^2 + m^2c^4$)

\begin{figure}[H]
    \centering
    \includegraphics{figures/lec03-08.png}
     \caption{}
\end{figure}

\subsection{Setup}
For a massless photon ($m = 0$):
\[
    E = pc
\]
And:
\[
    E = \frac{hc}{\lambda}
\]

So:
\begin{equation}
    p = \frac{h}{\lambda}
    \label{cscattere2}
\end{equation}

\subsection{Conservation and Relativity}
Conserving momentum (underlines omitted for speed):
\[
    p_1 = p_e + p_2
\]
\[
    p_e = p_1 - p_2
\]

Squaring both sides:
\[
    p_e^2 = p_1^2 + p_2^2 - 2p_1 \cdot p_2
\]
\begin{equation}
    p_e^2 = p_1^2 + p_2^2 - 2p_1p_2 \cos \theta
    \label{cscattere1}    
\end{equation}

And then by conservation of energy:
\[
    E_1 + E_e = E_2 + E_e'
\]

Where $E_1$ is the incoming photon energy, $E_e$ is the energy of the electron at rest in atom before collision, $E_2$ is the deflected photon energy and finally $E_e'$ is the deflected electron's energy.

Using this:
\[
    p_1c + m_e c^2 = p_2 c + \sqrt{p_e^2c^2 + m_e^2c^4}
\]
\[
    \implies p_1 - p_2 + m_ec = \sqrt{p_e^2 + m_e^2c^2}
\]

And squaring both sides:
\[
    (p_1 - p_2)^2 + \cancel{m_e^2c^2} + 2m_ec(p_1 - p_2) = p_e^2 + \cancel{m_e^2c^2}
\]

Substituting in Eqn \ref{cscattere1} for $p_e^2$
\begin{align*}
    (p_1 - p_2)^2 + 2m_ec(p_1 - p_2) &= p_e^2\\
    (p_1 - p_2)^2 + 2m_ec(p_1 - p_2) &= p_1^2 + p_2^2 - 2p_1p_2 \cos \theta
\end{align*}

And rearranging:
\begin{align*}
    (p_1 - p_2)^2 + 2m_ec(p_1 - p_2) &= p_1^2 + p_2^2 - 2p_1p_2 \cos \theta\\
    p_1^2 + p_2^2 - 2p_1p_2 + 2m_ec(p_1 - p_2) &= p_1^2 + p_2^2 - 2p_1p_2 \cos \theta\\
    -2p_1p_2 + 2m_ec(p_1 - p_2) &= - 2p_1p_2 \cos \theta\\
    -p_1p_2 + m_ec(p_1 - p_2) &= - p_1p_2 \cos \theta\\
    m_ec(p_1 - p_2) &= - p_1p_2 \cos \theta + p_1p_2\\
    m_ec(p_1 - p_2) &= p_1p_2 (1 - \cos \theta) 
\end{align*}

Substituting Eqn \ref{cscattere2}:
\begin{align*}
    m_ec(p_1 - p_2) &= p_1p_2 (1 - \cos \theta)\\
    m_ec\left(\frac{h}{\lambda_1} - \frac{h}{\lambda_2}\right) &= \frac{h}{\lambda_1} \frac{h}{\lambda_2} (1 - \cos \theta)\\
    m_ec\left(\frac{h}{\lambda_1} - \frac{h}{\lambda_2}\right) &= \frac{h^2}{\lambda_1 \lambda_2} (1 - \cos \theta)\\
    m_ec\left(\frac{1}{\lambda_1} - \frac{1}{\lambda_2}\right) &= \frac{h}{\lambda_1 \lambda_2} (1 - \cos \theta)\\
    m_ec\left(\frac{\lambda_1 \lambda_2}{\lambda_1} - \frac{\lambda_1 \lambda_2}{\lambda_2}\right) &= h (1 - \cos \theta)\\
    m_ec\left(\lambda_2 - \lambda_1\right) &= h (1 - \cos \theta)\\
    \left(\lambda_2 - \lambda_1\right) &= \frac{h}{m_ec} (1 - \cos \theta)
\end{align*}

Which is the Compton Equation. This shows that the change in wavelength is proportional to $1 - \cos \theta$.

\section{Conclusions}
The photoelectric effect and Compton scattering are two more physical phenomena that cannot be explained using traditional classical mechanics with EM waves alone. They both require assuming photons of energy $E = hf$ to be adequately explained.