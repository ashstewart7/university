% !TeX root = main.tex
\lecture{10}{Fri 05 Dec 2025 12:00}{A Quantum Mechanical Wave Equation}
In this lecture:
\begin{itemize}
    \item The Schr{\"o}dinger Equation.
    \item Expectation values of observables.
\end{itemize}

\section{Recap}
For a free particle, moving in the +'ve x direction, the wavefunction is given by:
\[
    \psi = A e^{ikx} e^{-i \omega t}
\]
Where the first e-term is the position dependence, and the second is the time dependence. We can separate these two terms.

\section{The Schrodinger Equation}
Note: This is not derivable from basic physics. It is a postulate, like the Bohr model, which we are confident is true because it's been empirically demonstrated.

We want to build a wave equation for a quantum mechanical free particle in a potential $V(x, t)$. This is a 1D potential with free movement in the x direction only. This wave equation is the Schrodinger Equation.

We assume the following:
\begin{itemize}
    \item Conservation of Energy applies, hence total energy is kinetic + potential energy. $T + V = E$.
    \item The equation is a linear differential equation. We need this, otherwise superposition would break, as this relies on two solutions to the wave equation being added together also forming a linear solution.
    \item This must not break the existing rules we've worked out, so $P = \frac{h}{\lambda} = \hbar k$ and $E = hf = \hbar \omega$
\end{itemize}

\subsection{Constructing the TISE}
Now we play a game of ``guess the terms'' to try to determine values for kinetic, potential and total energy. We can then substitute these into $E = T+V$. This is \textbf{not} a formal derivation, but is a motivated construction.

We start with our wave function:
\[
    \psi = A e^{ikx} e^{-i \omega t}
\]

We note that this has $k$ and $\omega$ present, so we have sufficient information to determine momentum and energy. We start by differentiating wrt $x$:
\[
    \frac{\partial}{\partial x} \psi(x, t) = ik A e^{ikx} e^{-i \omega t} = ik \psi
\]

This lets us extract an $ik$, but we want a $\hbar k$. Since $(i k)(-i \hbar) = \hbar k$, we can multiply by $-i \hbar$ to get $\hbar k$:
\[
    -i \hbar \frac{\partial}{\partial x} \psi = (-i \hbar)(ik) (\psi) = (\hbar k) \psi
\]

Crucially, this seems to let us extract $\hbar k$, and therefore find momentum. We're going to define the momentum operator in the x direction as:
\[
    \hat{p_x} = -i \hbar \frac{\partial}{\partial x}
\]
We can apply this operator to our wavefunction to determine x-directional momentum. Every observable will have an operator that we can apply to determine it.

We can write this as an eigenvalue equation (where the L.H.S is application of an operator, not multiplication):
\[
    \hat{p}_x [\psi] = p_x \psi
\]

This is defined as doing some operator onto a function, and getting back some multiple of the wavefunction, where that multiple is a useful quantity. This $p_x$ is the x-direction momentum and is a real number - we call this the eigenvalue. Note that, like matrices, operators are not commutative, so $\hat{p}_x \psi \not= \psi \hat{p}_x$.

It is important to note that the fact this works and we can get an eigenvalue tells us that momentum is a well defined quantity in this system, which is not always going to be true.

\subsection{Finding Kinetic Energy}
Now we have momentum, we can use the following to determine kinetic energy:
\[
    T = \frac{p^2}{2m}
\]
What if we therefore had:
\[
    \hat{T} = \frac{1}{2m} \hat{p} \hat{p} = - \frac{\hbar^2}{2m} \frac{\partial^2}{\partial x^2}
\]
Where the two $\hat{p}$s represent two subsequent applications of the momentum operator.

\subsection{Total Energy}
For total energy, we want to pull out $\omega$ as $E = \hbar \omega$. We therefore get:
\[
    \hat{E} = i \hbar \frac{\partial}{\partial t}
\]

\subsection{Potential Energy}
Potential Energy $V$ is totally general and can be ugly. We only consider simple constant potentials in QM1. This gives us a neat:
\[
    \hat{V} \psi = V \psi
\]
Where $V$ is a known and constant value for potential.

\subsection{Putting It All Together}
Now we have these operators, we can substitute them into $T + V = E$ to get the Schrodinger Equation.

\[
    \boxed{-\frac{\hbar^2}{2m} \frac{\partial^2 \psi(x, t)}{\partial x^2} + V(x, t) \psi(x,t) = i \hbar \frac{\partial \psi(x, t)}{\partial t}}
\]

When the potential is independent of time, we can factorise out the t dependence to get to the Time Independent Schrodinger Equation, TISE, (given in formula sheets). We begin by separating the time terms:

\[
        - \frac{\hbar^2}{2m}
        \frac{\partial^2 \psi(x)}{\partial x^2} \phi (t) + V(x) \psi(x) \phi(t)
        = i \hbar
        \frac{\partial \phi (t)}{\partial t}
        \psi(x)
\]


Dividing both sides by $\phi(t) \psi(x)$:
\[
\underbrace{\frac{1}{\psi(x)} \left[ -\frac{\hbar^2}{2m} \frac{d^2 \psi(x)}{dx^2} + V(x)\psi(x) \right]}_{\text{spatially dependent}} 
= \underbrace{\frac{i\hbar}{\phi(t)} \frac{d \phi(t)}{dt}}_{\text{time dependent}}
\]
Since the L.H.S is purely spatial, and the R.H.S purely time, they are entirely independent. It is only true that these are equal for all points in space and time if these are equal to a constant. We call this constant $E$.

\[
    \frac{1}{\psi(x)} \left[ -\frac{\hbar^2}{2m} \frac{d^2 \psi(x)}{dx^2} + V(x)\psi(x) \right] = E
\]
\[
    -\frac{\hbar^2}{2m} \frac{d^2 \psi(x)}{dx^2} + V(x)\psi(x) = E\psi(x)
\]


Or we can rewrite this as:
\[
    \left[\frac{- \hbar^2}{2m} \frac{d^2}{dx^2} + V(x)\right] \psi = E \psi
\]
The contents of the brackets form an operator. We call this the ``Hamiltonian'':
\[
    \hat{H} [\psi] = E \psi
\]

\section{Expectation Values}
Expectation values represent the average of an observable. It is denoted with angle brackets, for example:
\[
    \langle p_x\rangle
\]
Is the average value for x-momentum if measured many times. In order to get this, we sandwich the operator between $\psi^*$ and $\psi$ and we integrate. For example, the operator for $\hat{x}$ is just $= x$. Therefore:
\[
    \langle x\rangle = \int_{-\infty}^{\infty} \psi^* x \psi \, dx
\]

Going back to our infinite well example:
\[
    \psi_1 = \sqrt{\frac{2}{L}} \sin kx, \qquad k = \pi / L
\]
\[
    \langle x\rangle = \frac{2}{L} \int_{0}^{L} x \sin^2 \left(\frac{\pi}{L} x\right) \, dx
\]
\[
    =\frac{2}{2L} \int_{0}^{L} x \left(1 - \cos\left(\frac{2 \pi x}{L}\right)\right) \, dx
\]

Carrying out the integral by parts, we get:
\[
    \langle x\rangle = \frac{L}{2}
\]
Which is physically sensible, we expect the particle to lie, on average, in the middle. Note that this is an average. If we had a particle at $n = 2$, the standing wave created would have a node at $L/2$. This means it would be impossible to find the particle in the middle, but we can still have this as our expectation value.

\section{In Conclusion}
\begin{itemize}
    \item Operators pull out observables from the wavefunction.
    \item We build operators for kinetic (T), potential (V) and total energy (E) and substitute into $T + V = E$ to get the Schrodinger Equation.
    \item Expectation values give us average values of an operator. This lets us find average values for things without well defined eigenvalues.
\end{itemize}


