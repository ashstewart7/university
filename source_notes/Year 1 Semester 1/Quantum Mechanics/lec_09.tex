% !TeX root = main.tex
\lecture{9}{Fri 28 Nov 2025 12:00}{Wavefunctions for Quantum Particles}

In this lecture:
\begin{itemize}
    \item Reminder and recap of the classical wave equation, building to the complex quantum mechanical version.
    \item The idea of probability density and probability amplitude.
    \item The idea of standing waves and the ``particle in a box'' being modelled as one.
\end{itemize}

\section{Recap of Classical Waves}
The classical wave equation is:
\begin{equation}
    \frac{\partial^2E}{\partial x^2} = \frac{1}{c^2}\frac{\partial^2E}{\partial t^2}
\end{equation}

Solutions to this have the form:
\[
    E(x, t) = E\sin(kx - \omega t)
\]
\[
    E(x, t) = E \cos(kx - \omega t)
\]

As the wave equation is linear, linear sums of individual solutions form a solution. I.e. the summation of two different wave functions in this form is also a wave function. Recall that $k = 2 \pi / \lambda$ ``the wave number'' and $\omega = 2 \pi f$ ``angular frequency''.

\subsection{Direction}
Wavefunctions with $kx - \omega t$ move to the right (in the positive x direction).

Wavefunctions with $kx + \omega t$ move to the left (in the negative x direction).

\subsection{Phase}
Consider some point on a wave. Each point has a phase $\phi$, given by $\sin \phi = \sin kx-\omega t$. The phase of a point is constant, regardless of propagation.
\begin{equation}
    \phi = kx - \omega t
\end{equation}

For a constant point, i.e with phase $\pi / 2$, the phase must be constant. Hence, as time increases, x must also increase, and hence the wave must be moving rightwards along the positive x-axis, as required.

To verify this, we can differentiate wrt time, considering a point where phi does not change, so the derivative is zero:
\[
    \frac{\partial \phi}{\partial t} = k \frac{\partial x}{\partial t} - \omega \frac{\partial t}{\partial t} = 0
\]
\[
    k \frac{\partial x}{\partial t} = \omega
\]
\[
    \frac{\partial x}{\partial t} = \frac{\omega}{k} > 0 \qquad \text{ so travelling right.}
\]

This also happens to prove the wave speed equation again.

\subsection{Things Get Complex}
It is important to note that we are often dealing with complex representations, that we'll need when things become quantum-y. We can also write a solution to the wave equation as:
\[
    E(x, t) = \mathfrak{Re}\left\{E e^{i(kx - \omega t)}\right\}
\]

This doesn't change anything, it's just a convenient repacking of the trigonometric terms into a slightly nicer form - it's just a maths trick. Even if not stated, there is always an implicit ``take the real part'' at the end, to get the actual cos() etc terms and real numbers we care about.

\section{Quantum Mechanical Wave Function}
In QM, the biggest change is that the particle wavefunction \textbf{truly is complex}. We do not take the real part at the end, and represents something which is fundamentally a complex number. Therefore, the wave function is:
\[
    \boxed{\psi(x, t) = A e^{i(kx - \omega t)}}
\]

Again, note that both $\psi(x, t)$ and $A$ are complex numbers. We can rewrite $A$ as $A = A' e^i$, where $A'$ is some real number.

We can now split up the terms:
\[
    \psi = Ae^{ikx}e^{-i \omega t}
\]

Where the first term is the spatial function $\psi(x)$ and the second term is the time dependence (or phase) $\phi(t)$. This is the wavefunction for a particle moving in the positive x, while something moving in the negative x would be:
\[
    \psi = Ae^{-ikx}e^{-i \omega t}
\]

Note that this is just conventions, and direction could instead be bundled into the $A$ term, but we ignore this here.

\section{What Does This Really Mean?}
The amplitude of the quantum wavefunction for a particle represents the probability \emph{amplitude}. This is a complex number, so has a magnitude and a phase. This phase is what gives us interference.

We physically cannot measure the phase of a particle such as an electron or a photon - it is not an observable. We can compare the phase difference between two particles, and one phase relative to another, but we cannot measure absolute phase in QM.

\subsection{Observables}
Something which we physically can measure is called an observable. They must be real numbers only. Something which is not an observable physically cannot be measured, regardless of how good the equipment is etc. In QM, this means we take complex conjugates to get to an observable.

Recall:
\[
    z = a + ib \qquad z^* = a - ib
\]
Then:
\[
    z*z = a^2 + b^2 = | z |^2
\]
Which is real. This gives us \emph{probability density}, which is something we can actually measure.

\[
    P(x) = \psi^*(x) \psi(x) = | \psi |^2
\]

This gives us the probability density function for finding this particle at some position $x$. Note that we have to integrate over a range of $x$ values to get an actual probability from this probability density.

The probability of finding the particle at some exact precise value $x = a$, where a is some real number is precisely zero. We must integrate over a range, and get a probability for this range - as $x$ can take any continuous value, the probability of finding the particle at any one (of infinitely many) is infinitely small.   This is also forbidden by the uncertainty principle (Lec 08).


\section{Particle in a Box}
Standing waves become a key concept for a particle constrained in a box. A standing wave is made up of two parts:
\begin{itemize}
    \item A wave propagating in one direction: $e^{ikx}$.
    \item A reflected wave propagating in the opposite direction: $e^{-ikx}$
\end{itemize}

We know that we can add the superposition of two different solutions to the wave equation, to give another solution, so:
\[
    \psi = A_1 e^{ikx}e^{-i \omega t} + A_2 e^{-ikx}e^{-i \omega t}
\]

Since the box is symmetrical, $A_1 = -A_2$ (a positive equality works too, we just get a cosine rather than a sine).

\[
    \psi = A_1 \left(e^{ikx} - e^{-ikx}\right)e^{-i \omega t}
\]
\[
    = A_1\left(\left[\cos(kx) + i \sin(kx)\right] - \left[\cos(-kx) + i \sin(-kx)\right]\right) e^{-i \omega t}
\]
\[
    = A_1\left(\left[\cos(kx) + i \sin(kx)\right] - \left[\cos(kx) - i \sin(kx)\right]\right) e^{-i \omega t}
\]
\[
    = A_1\left(\left[i \sin(kx)\right] - \left[- i \sin(kx)\right]\right) e^{-i \omega t} 
\]
\[
    = A_12i \sin(kx) e^{-i \omega t} 
\]
\[
    \boxed{\psi = A \sin (kx) e^{-i \omega t} \qquad \text{where: } A = A_1 2i}
\]

And using this to find probability density:
\[
    P(x) = \psi^* \psi = A^*A \sin(kx) \sin(kx) e^{-i \omega t} e^{i \omega t}
\]
\[
    = |A|^2 \sin^2(kx)
\]

In the last lecture, we related $k$ to the length of the box, and the energy state of the particle trapped in it. Using this, and considering the $n = 1$ energy level:
\[
    P(x) = |A|^2 \sin^2 \frac{\pi x}{L}
\]

Note that in many situations, albeit not this one, $P(x)$ may be time dependent. $P(x)$ is also not uniform. Classically, we would expect a uniform probability for the particle's location equal at all points. However, using this QM wavefunction, we have a much greater probability of finding the particle at the centre of the box than the outer regions.

\begin{figure}[H]
    \centering
    \includegraphics[width=0.75\textwidth]{figures/lec09-01.png}
     \caption{}
\end{figure}

\subsection{Normalisation}
Previously, we've mostly ignored the value of $A$. However, in order to get numerical values out, we need to find a value for $|A|^2$. Since $P(x)$ is a probability distribution, the integral of the p.d.f. from negative to positive infinity must be one (i.e. the particle is trapped in the box, and must be \emph{somewhere}).

\[
    \int_{-\infty}^{\infty} P(x) \, dx = \int_{0}^{L} |A|^2 \sin^2 \frac{\pi x}{L} \, dx = 1
\]
We can change the bounds as we know the particle must be in the box somewhere, so the probability of finding it outside of the box is zero.
\[
= |A|^2 \int_{0}^{L} \frac{1 - \cos\!\left(\frac{2\pi x}{L}\right)}{2}\,dx
\]

\[
= \frac{|A|^2}{2} \int_{0}^{L}
\left(1 - \cos\!\left(\frac{2\pi x}{L}\right)\right)\,dx
\]

\[
\frac{|A|^2}{2}
\left[
x - \frac{L}{2\pi}
\sin\!\left(\frac{2\pi x}{L}\right)
\right]_0^L
= 1
\]

\[
\frac{|A|^2}{2}
\left(L - 0\right)
= 1
\]

\[
|A|^2 \frac{L}{2} = 1
\]

\[
\boxed{
|A|^2 = \frac{2}{L}
}
\]

Therefore:
\[
    P(x) = \frac{2}{L} \sin^2 kx \qquad \text{for } n = 1
\]

Note that this value of the prefactor $A$ is true only for this geometry of problem. 

\subsection{Using This!}
Now we have no unknown prefactors, we can find actual probabilities. For example, the probability of the particle being in the right half of the box is given by:

\[
    \int_{L/2}^{\infty} P(x) \, dx = \int_{L/2}^{L} \frac{2}{L} \sin^2 kx \, dx
\]
\[
    = \frac{2}{2L} \int_{L/2}^{L} 1 - \cos(2kx) \, dx
\]
\[
    = \frac{1}{L} \left[ x - \frac{\sin 2 kx}{2k}\right]^L_{L/2}
\]

And using $k = 2 \pi / \lambda = 2 \pi / 2 L = \pi / L$ (for the $n = 1$) energy level, per Lec 08:
\[
    = \frac{1}{L} \left[ x - \frac{\sin \frac{2 \pi x}{L}}{\frac{2 \pi}{L}}\right]^L_{L/2}
\]
\[
    = \frac{1}{L}\left(\left[L - \frac{\sin \frac{2 \pi L}{L}}{2 \pi / L}\right] - \left[\frac{L}{2} - \frac{\sin\frac{2 \pi L}{2 L}}{2 \pi / L}\right]\right)
\]
\[
    = \frac{1}{L}\left(\left[L - \frac{\sin 2\pi}{2 \pi / L}\right] - \left[\frac{L}{2} - \frac{\sin\pi}{2 \pi / L}\right]\right)
\]
\[
    = \frac{1}{L}\left(L - \frac{L}{2}\right)
\]
\[
    = \frac{1}{2}
\]

This makes physical sense, as we're considering half the box, and the box (and the p.d.f. for the particle) are symmetrical. 

