% !TeX root = main.tex
\lecture{S7}{Wed 22 Oct 2025 12:00}{Fitting a Straight Line 1}
We want to create a model for a straight line:
\[
    M(x, \theta) = mx + c
\]

Where are datapoints are given by this model and some additive noise:
\[
    D = M(x, \theta) + \epsilon, \quad \epsilon \sim \mathcal{N}(0, \sigma_i)
\]

The general recipe for line fitting is given by:
\begin{enumerate}
    \item A generative model for the data, with knowledge of how the noise is distributed.
    \item Likelihood function.
    \item A method for finding the maximum likelihood.
    \item Method for finding the uncertainties on best fit parameters.
    \item A method for checking how good the fit is.
\end{enumerate}

We can write down the likelihood function for this model as:

\begin{align*}
    P(D_i \mid \theta) &= \frac{1}{\sigma_{D_i} \sqrt{2 \pi}} \exp\left(\frac{-(D_i - M(x_i, \theta))^2}{2 \sigma^2_{D_i}}\right)\\
    P(D \mid \theta) &= \prod_{i=1}^{n} P(D_i \mid \theta)
\end{align*}

And again:
\[
    \LL = \ln P(D \mid \theta) = \sum_{i=1}^{n} \ln P(D_i \mid \theta) \propto \sum_{i=1}^{n} \left(\frac{-(D_i - M(x_i, \theta))^2}{2\sigma^2_{D_i}}\right)
\]

We want to find the parameters of distribution that maximise the (log)likelihood:
\[
    \hat{\theta} = \arg\max_{\theta} P(D\mid\theta)
\]
Or:
\[
    \hat{\theta} = \arg\max_{\theta} \LL
\]

There are a number of different approaches to do this:
\begin{itemize}
    \item Find where all first derivatives equal zero (as last lecture, various clever algorithms to do so).
    \item Brute force on a grid.
    \item Iterative or stochastic methods.
    \item Analytic maximisation for a simple linear model - see next lecture.
\end{itemize}

\section{Finding Maximum Likelihood}
The most crude way to do this is to build a grid of all values of $m$ and $c$, and iterate through over all points (with some resolution) to find the maximum likelihood generated by them. Plotting M (as colour):
\begin{figure}[H]
    \centering
    \includegraphics[width=0.75\textwidth]{figures/lec07-01.png}
     \caption{}
\end{figure}

And the likelihood:
\begin{figure}[H]
    \centering
    \includegraphics[width=0.75\textwidth]{figures/lec07-02.png}
     \caption{}
\end{figure}

We assume that the grid point with the highest likelihood and the true point with the highest likelihood are the same. In this case, the grid resolution is small enough that this is true, but it may not always be. This gives:
\begin{align*}
m &= 2.0552763819095476\\
c &= -0.9447236180904524
\end{align*}

\begin{figure}[H]
    \centering
    \includegraphics[width=0.75\textwidth]{figures/lec07-03.png}
     \caption{}
\end{figure}

We see that this does a good, but not perfect job, of fitting the data. This is due to noise in the data, and is acceptable, provided it's within a reasonable uncertainty.