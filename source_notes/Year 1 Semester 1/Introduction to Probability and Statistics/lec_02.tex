% !TeX root = main.tex
\lecture{2}{Thu 02 Oct 2025 09:00}{Population Statistics}

\subsection{Accuracy and Precision}
We usually take measurements to determine some kind of true value. Usually, we can't actually know what this true value is, but if we could there are two bits of terminology that is particularly important:

    \textbf{Accuracy:}
    Accuracy is the `closeness' between our value and the `true' value.

    \textbf{Precision:}
    Precision is the `closeness' between our measurements, i.e. how spread out are our various measurements.


\subsection{Error}
\textbf{Random Error:} is uncertainty related to the fact that our measurements are only a finite sample, so is not going to be immediately representative of the true value. The smaller this error, the more precise the measurement is.

\textbf{Systematic Error:} is related to some kind of issue with the measurement or the equipment. This shifts all values, and negatively affects accuracy (but leaves precision unchanged)

Taking many repeat measurements decreases the effects of random error, but the effects of systematic error are much harder to combat.

Ideally, we want to be both precise and accurate, however accuracy is arguably more important. This is because a value which is precise, but not accurate may lead to false conclusions around the inaccurate value.