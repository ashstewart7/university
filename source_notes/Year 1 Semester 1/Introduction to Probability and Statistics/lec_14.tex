% !TeX root = main.tex
\lecture{3}{Thu 14 Nov 2025 11:00}{Combining Probabilities}

Today we will arrive at:
\begin{itemize}
    \item The formula for $P(A \cap B)$ (Probability of A and B).
    \item Summing mutually exclusive events.
\end{itemize}

\section{More Set Theory}
We have a sample space $\Omega$, and a subset labelled $A$. We then have the remainder of $\Omega$ (the portion of $\Omega$ which is not in $A$), denoted "A Complement" - $A^C$ or $\bar{A}$. We also have a subset labelled $B$. 

We can define A using set builder notation, to slightly redundantly say "A is the set of all x'es which are in A":
\[
    A = \left\{x \mid x \in A\right\}
\]

There is some overlap between A and B. We denote this intersection as $A \cap B$.
\[
    A \cap B = \left\{x \mid x \in A \text{ and } x \in B\right\}
\]

Everything written in A or B (including the intersection) is called the union, $A \cup B$:
\[
    A \cup B = \left\{x \mid x \in A \text{ or } x \in B\right\}
\]
Note that ``or'' in standard language excludes both, i.e. you may have x or you may have y. In mathematics, we refer to this as XOR (exclusive or). ``Or'' by itself does allow for this case of both, so an item in A or B may be in A alone, B alone, or both (i.e. in the intersection).

We also have the empty set $\emptyset = \{\}$. If two sets have no common elements, the intersection is this empty set. We say that the events are mutually exclusive (they cannot both happen) and the sets are pairwise disjoint. The empty set is the complement of $\Omega$, $\emptyset = \Omega^C$.

\section{De Morgan's Laws}
De Morgan's Laws give us these relations:
\begin{equation}
    (A \cup B)^C = A^C \cap B^C
\end{equation}

\begin{equation}
    (A \cap B)^C = A^C \cup B^C
\end{equation}

This can be illustrated visually as follows:
\begin{figure}[H]
    \centering
    \includegraphics[width=0.75\textwidth]{figures/lec14-02.png}
     \caption{}
\end{figure}

\begin{figure}[H]
    \centering
    \includegraphics[width=0.75\textwidth]{figures/lec14-03.png}
     \caption{}
\end{figure}

\section{The Inclusion-Exclusion Principle}
The number of elements in A and B is given by:
\[
    |A \cup B| = |A| + |B| - |A \cap B|
\]
The last term is required to account for the intersection of A and B being included in A, and included in B. Therefore it double-counts the intersection, and we subtract it away.

The same is true of probability functions:
\[
    P(A) + P(B) = P(A \cup B) + P(A \cap B)
\]
In other words, the probability of A + the probability of B is the probability of A or B + the probability of A + B.

This has the following consequences:
\[
    P(\emptyset) = 0
\]
As:
\[
    P(A) = \frac{|A|}{|\Omega|} \implies P(\emptyset) = \frac{0}{|\Omega|} = 0
\]

And:
\[
    P(A) = p \implies P(A^C) = 1 - p
\]
As:
\[
    \Omega = A \cup A^C
\]
\[
    P(\Omega) = P(A) + P(A^C) - P(A \cup A^C) \qquad \text{By inclusion-exclusion principle.}
\]
\[
    P(A \cup A^C) = P(\emptyset) = 0
\]
\[
    1 = P(A) + P(A^C) \qquad \text{As: } P(\Omega) = 1
\]

\section{Multiple Events}
Given some events $e_n$, the inclusion-exclusion principle says:
\[
    P(e_1 \cup e_1) = P(e_1) + P(e_2) - P(e_1 \cap e_2)
\]
Hence for independent events ($e_1 \cap e_2 = \emptyset$), the probability of $e_1$ or $e_2$ occurring is:
\[
    P(e_1 \cup e_2) = P(e_1) + P(e_2)
\]

\subsection{What about 3 events?}
\begin{figure}[H]
    \centering
    \includegraphics[width=0.75\textwidth]{figures/lec14-04.png}
     \caption{}
\end{figure}
For three events, our venn diagram becomes more complex. We want to calculate $|e_1 \cup e_2 \cup e_3|$ (in this general example, which circle is which event is irrelevant).

This is given by:
\[
    |e_1 \cup e_2 \cup e_3| = |e_1| + |e_2| + |e_3| - |e_1 \cap e_2| - |e_1 \cap e_3| - |e_2 \cap e_3| + |e_1 \cap e_2 \cap e_3|
\]

Note the final term, as the central portion is included three times when summing the whole event, but subtracted three times when removing intersections, so we must add it back.

If they are all pairwise disjoint, so $e_i \cup e_j = \emptyset$, then:
\[
P(e_1 \cup e_2 \cup \cdots \cup e_3) = P(e_1) + P(e_2) + \cdots + P(e_n)
\]
Or:
\[
P \left(\bigcup^N_{n=1}e_n\right) = \sum_{n=1}^{N} P(e_n)
\]

\subsection{Normalisation}
If the events are all mutually exclusive, and the sample space is ``tiled'' by the events (i.e. one of them must happen), then:
\[
    P(\Omega) = P(e_1) + P(e_2) + \cdots + P(e_n) = 1
\]

