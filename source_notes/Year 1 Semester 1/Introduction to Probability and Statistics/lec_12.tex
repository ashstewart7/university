% !TeX root = main.tex
\lecture{P1}{Fri 07 Nov 2025 11:00}{\textbf{Start of Probability}: Introduction}
\paragraph{What is probability?} Probability is the pure mathematical description of randomness.

\section{Empirical Probability}
Say we want to group trees into four sets:
\begin{itemize}
    \item Tall, or not.
    \item Variegated (has a lighter coloured leaf border) or not.
\end{itemize}

In a park of 142 trees, we observe:
\begin{figure}[H]
    \centering
    \includegraphics[width=0.75\textwidth]{figures/lec12-01.png}
     \caption{}
\end{figure}

We denote Tall as $T$, Variegated as $V$ and not Tall/not Variegated as $\bar{T}$ and $\bar{V}$. We use $n$ as the number of number of trees that satisfy the parameters, i.e $n(T, \bar{V}) = 72$. $N = 142$ is the total number of trees, so:
\[
    N = n(T, V) + n(T, \bar{V}) + n(\bar{T}, \bar{V}) + n(\bar{T}, V) 
\]


Similarly, we define the fraction of trees that meet the provided criteria as $f$, i.e:
\[
    f(T, V) \equiv \frac{n(T, V)}{N}
\]
Hence:
\[
    1 = f(T, V) + f(T, \bar{V}) + f(\bar{T}, \bar{V}) + f(\bar{T}, V) 
\]

We define probability as the limit of this as $N \to \infty$:
\[
    \boxed{P(T, V) = \lim_{N \to \infty} f(T, V) = \lim_{N \to \infty} \frac{n(T, V)}{N}}
\]
And discrete probability as the limiting fraction of the times an event will occur. Probability is bounded between 0 (the event \textbf{never} occurs) and 1 (the event \textbf{always} occurs). In a particular experiment, the total probability is one, therefore something must happen (we just may not know what exactly).

\section{Set Theory}
A set is a collection of elements, i.e.:
\[
    A = \{1, 2, 3\}
\]
A deck of cards is a set, with 52 elements. Elements can be anything, including other sets.

\subsection{Subsets}
Consider some set $A$. Another set, $B$ is a ``subset'' of $A$ if it contains (solely) some of the elements of A. We denote this $B \subset A$.

For example:
\[
    A = \{1, 2, 3, 4, 5, 6\}\]\[
    B= \{1, 2, 3\}\]\[
    C = \{4, 5, 6\}\]\[
    D = \{1, 5, 7\}\]
\[
    B \subset A
\]
\[
    C \subset A
\]
\[
    D \not\subset A
\]

\subsection{Sample Space}
We observe a particular outcome, and the set of all possible outcomes we could observe is called the sample space $\Omega$. Any event (that can occur) is a subset of $\Omega$.

If drawing 5 cards from a deck, all possible combinations of 5 cards form the sample space. Drawing an ace is an event.

\section{Probability Functions}
Probability assigns a value to every event in $\Omega$ to quantify how likely it is to happen. This value is the probability of the event. 

If something must happen, then we have:
\[
    P(\Omega) = 1
\]
This is called `normalisation'.

