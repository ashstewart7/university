% !TeX root = main.tex
\lecture{7}{Thu 22 Oct 2025 11:00}{Energy and Power of Waves}

\section{Kinetic Energy for a Sine Wave}
Assuming a sine wave created by a harmonic oscillator. We can consider every particle on this string as acting as its own harmonic oscillator. We can determine the kinetic energy for one single harmonic oscillator (of an infinitesimally small length of string)

\[
    KE = \frac{1}{2}mv^2
\]
\[
    (KE)_\text{max} = \frac{1}{2}m v_\text{max}^2
\]
Mass is given by the density of the string per unit length multiplied by the length of the section of string: $\mu dx$, and the maximum velocity by $\omega A$:
\[
    (KE)_\text{max} = \frac{1}{2}(\mu dx)(\omega A)^2
\]

This gives us the kinetic energy of a single oscillator. We now want to get the total energy across a single wavelength, by integrating:
\[
    E = \int_{0}^{\lambda} \frac{1}{2} \omega^2 A^2 \, \mu dx
\]
\[
    E = \frac{1}{2} \omega^2 A^2 \mu \int_{0}^{\lambda}  \, dx
\]
\[
    E = \frac{1}{2} \omega^2 A^2 \mu \lambda
\]

Note that strictly speaking, at any point half the wave's energy is kinetic and half is potential. Instead of calculating potential energy, we consider a singular oscillator. In this single oscillator, the two quantities oscillate in opposition to each other, so max KE means zero PE. We can say $E_\text{total} = KE_\text{max}$ therefore, for a singular oscillator. This works out when we integrate across all oscillators.

And for power:
\[
    \text{Power} = \frac{\text{Energy}}{\text{Time}}
\]
\[
    P = \frac{\frac{1}{2}(\omega A)^2 \mu \lambda}{T}
\]
\[
    = \frac{1}{2} (\omega A)^2 \mu v
\]

\section{Standing Waves}
Considering a standing wave with $\lambda = 2L$ (hence a single loop with two nodes at each boundary and one antinode in the middle):

\[
    y_t = y_1 + y_2
\]
\[
    \implies (KE)_t = (KE)_1 + (KE)_2
\]

Where $y_1$ and $y_2$ are identical except for their direction, but they carry the same kinetic energy. The KE of a standing wave is the sum of the KE of two waves that make it up.

Note: Normally, A is the amplitude of the travelling wave (hence 2A is the amplitude of the standing wave created by them), however it is sometimes ambiguous what is being referred to by A - some questions may give the standing wave amplitude and also denote this as A.

\section{Interference}
\textbf{Superposition}: When waves overlap in the same region, the resulting wave is the algebraic sum of waves (they interfere)

Consider two waves:
\[
    y_1 = A \cos(kx - \omega t) \qquad y_2 = A \cos(kx - \omega t + \delta)
\]
Where both waves are travelling in the same direction with the same amplitude, but the second wave has some phase shift $\delta$.

We can sum the waves as with standing waves:
\[
    y = A \cos(\underbrace{kx - \omega t}_\alpha) + A \cos(\underbrace{kx - \omega t + \delta}_\beta)
\]
\[
    \text{Using:} \quad \cos \alpha + \cos \beta = 2 \cos \frac{\alpha + \beta}{2}\cos \frac{\alpha - \beta}{2}
\]
\[
    y = 2A \cos \left(\frac{\delta}{2} \right)\cos \left(kx - \omega t + \frac{\delta}{2}\right)
\]

So the amplitude of the new travelling (not standing) wave is dependant on delta, where amplitude is given by $2A \cos(\delta / 2)$.

Here are three examples of phase difference where $y_1$ is the blue line, $y_2$ is green and $y_\text{total}$ is red:
\begin{figure}[H]
    \centering
    \includegraphics[width=\textwidth]{figures/lec07-01.png}
     \caption{}
\end{figure}

Given two arbitrary waves, differences in phase can arise from either a difference in angular frequency, and/or in x.

\subsection{Example}
Suppose we have two sound sources, $S_1, S_2$. They oscillate in phase and emit harmonic waves of equal frequency and wavelength. Lets consider some point $P$ where the waves interact where the path differences are different (i.e. length from S1 to P $\not =$ length from S2 to P).

If there is an integer number of wavelengths in path difference, the interference is totally constructive, if path difference is equal to a half number of wavelengths (i.e. $1.5, 2.5, 3.5$) then the interference is totally destructive. 

From $S_1$:
\[
    y_1(l_1, t) = A \cos(k l_1 - \omega t)
\]

From $S_2$:
\[
    y_2(l_2, t) = A \cos(k l_2 - \omega t)
\]

Giving us a phase difference of:
\[
    \delta = (k l_1 - \omega t) - (k l_2 - \omega t) = k(l_1 - l_2)
\]
\[
    \delta = \frac{2 \pi}{\lambda} \Delta l
\]
Hence the signal strength at point P is dependant on the path difference.


\section{Beats}
Lets consider interference from waves with slightly different frequencies (but the same amplitude). We again have two sources $S_1$ and $S_2$. We observe the resultant amplitude after these have interfered at some value of $x$ (say $x = 0$ so the $kx$ term is $0$), where both sources are the same distance from this point, $P$.
\[
    p_1 = A \cos \omega_1 t
\]
\[
    p_2 = A \cos \omega_2 t
\]
\[
    p = p_1 + p_2 = A \left(\cos \omega_1 t + \cos \omega_2 t\right)
\]
\[
    p = 2A \cos \left(\frac{\omega_1 - \omega_2}{2}t\right) \cos \left(\frac{\omega_1 + \omega_2}{2}t\right)
\]

We also note the average value of omega as $\omega_\text{avg} = (\omega_1 + \omega_2) / 2$ and the variation from this value as $\Delta \omega = (\omega_1 - \omega_2) / 2$

\[
    p = 2A \cos(\Delta \omega t) \cos(\omega_\text{avg} t)
\]

\begin{figure}[H]
    \centering
    \includegraphics[width=0.75\textwidth]{figures/lec07-02.png}
    \caption{The resultant wave \textcolor{green}{$p$} in green, with \textcolor{red}{$p_1$} and \textcolor{blue}{$p_2$} in red and blue.}
\end{figure}

Note that (as the difference in frequency is small) the two waves start mostly in phase, leading to a high amplitude resultant wave. As they drift out of phase, the resultant amplitude drops (becoming zero when in antiphase). Phase difference is periodic, so they return in phase and this repeats, creating a cosine wave with oscillating amplitude.

We can consider the `envelope' that contains the resultant wave (given in purple):

The amplitude oscillates with angular frequency:
\[
    \omega = \left(\frac{\omega_1 - \omega_2}{2}\right) = 2 \pi f \implies f = \left(\frac{\omega_1 - \omega_2}{4 \pi}\right)
\]

The human ear cannot really tell the difference between the green signal and the purple envelope signal, and we have a perceived beat frequency of:
\[
    f_\text{beat} = 2\left(\frac{\omega_1 - \omega_2}{4 \pi}\right) = \frac{\omega_1}{2 \pi} - \frac{\omega_2}{2 \pi} = | f_1 - f_2|
\]
The 2 arises as our ears don't care about the difference between a positive and a negative amplitude. The cosine wave goes positive, zero, negative, zero (one oscillation) while the human ear perceives loud, quiet, loud, quiet (two oscillations).

Note: This holds up when the amplitude of the two waves is the same. If the amplitudes are different, we would have to re-derive to take this into account.