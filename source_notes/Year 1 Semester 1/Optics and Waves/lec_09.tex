% !TeX root = main.tex
\lecture{9}{Tue 29 Oct 2025 12:00}{Doppler Shift, Shockwaves and Optics Part One}

\paragraph{Doppler Shift:} If a wave emitter and observer are moving relative to each other, the observed frequency will be different to the emitted frequency. As wave emitting objects moves towards an observer, the perceived frequency of the wave increases. As the object moves away, the frequency decreases.

We consider two scenarios: (1) a moving source, and (2) a moving receiver, where the other is stationary. We will consider this for \textbf{sound waves}.

\section{Case 1: Moving Source, Stationary Receiver}
The source has frequency $f_s$ and period $T_s = 1/f_s$. It has velocity $u_s$, and the velocity of waves in the medium is $v$.

At $t=0$: The wave source creates a sound wave pulse, which begins to propagate out in all directions.

At $t = T_s$:
The initial sound wave has moved by $d = st$, so distance travelled by the wave is $v T_s$ distance units. The source has moved to the right by $u_s T_s$ distance units.
\begin{figure}[H]
    \centering
    \includegraphics[width=0.5\textwidth]{figures/lec09-1.png}
     \caption{}
\end{figure}

As this continues, we get something which looks like this:
\begin{figure}[H]
    \centering
    \includegraphics[width=0.3\textwidth]{figures/lec09-02.png}
     \caption{}
\end{figure}

This effectively gives us a wave which has two different frequencies and two different wavelengths, one in the forward direction and one each in the backwards.

We denote the distance between the source's new location and the wavelet in the forward direction as $\lambda_f$, and in the reverse direction as $\lambda_b$.

\[
    \lambda_f = vT_s - u_sT_s
\]
And the new frequency in the forward direction:
\[
    f_f = \frac{v}{\lambda_f}
\]
\[
    f_f = \frac{v}{(v - u_s)T_s}
\]
\[
    f_f = \frac{v}{(v - u_s)(1 / f_s)}
\]
\[
    f_f = f_s \frac{v}{v - u_s}
\]

We can do the same analysis considering the wavelength in the backwards direction:
\[
    f_b = f_s \frac{v}{v + u_s}
\]



\section{Case 2: Stationary Source, Moving Receiver}
Consider a periodic wave with wavelength $\lambda$ and speed $v$. Lets look at the case where the receiver is moving towards the source at a speed $u_r$. 

\begin{figure}[H]
    \centering
    \includegraphics[width=0.75\textwidth]{figures/lec09-03.png}
     \caption{}
\end{figure}

The receiver perceives the wave as moving at speed $v + u_r$ \footnote{This does not take special relativity into account, so no ultrafast receivers here, and since it's a sound wave, $v \ll c$.}. It therefore observes the frequency:
\[
    f_f = \frac{v + u_r}{\lambda}
\]

And if the receiver is moving in the opposite direction:
\[
    f_b = \frac{v - u_r}{\lambda}
\]

Note that as the source is not moving, the wavelength remains constant.
\section{Case 3: Both Source \& Receiver Moving}
In summary, for a moving source:
\[
    f_r = \frac{v}{v \pm u_s} f_s \quad (- \text{ towards, } + \text{ away.})
\]

And for a moving receiver:
\[
    f_s = \frac{v\pm u_r}{v} f_r \quad (+ \text{ towards, } - \text{ away.})
\]


We can combine these two equations to get the general form:
\[
    f_r = \frac{v \pm u_r}{v \pm u_s} f_s
\]

Note that the two $\pm$s are independent and the numerator/denominator values for the sign should be chosen according to the sign conventions for the numerator/denominator stated in the two individual equations. If moving towards, frequency increases. If moving away, frequency decreases.

\subsection{What about light waves?}
The doppler shift in frequency depends on the source/receiver speed relative to the medium. For light, we do not consider a traditional medium. We therefore cannot talk about absolute motion, we can only consider relative motion between the source and the receiver.

We have:
\[
    f_r = f_s \sqrt{\frac{c \pm u}{c \mp u}}
\]
If we expand this and ignore terms in $u^2/c^2$ (as they'll be tiny). We get:
\[
    \frac{\Delta f}{f_s} = \pm \frac{u}{c} \qquad \text{where: } \Delta f = f_r - f_s
\]

This only holds for $u \ll c$, as if this isn't true the  $u^2/c^2$ terms will be non-negligible. The square root appears due to time dilation. 

\begin{itemize}
    \item \textbf{Blueshift ($+$):} Source approaching (frequency increases).
    \item \textbf{Redshift ($-$):} Source receding (frequency decreases).
\end{itemize}

\section{Shock Waves}
Shock waves arise when the velocity of the source is faster than the velocity of waves in the medium. We have previously implicitly assumed $u < v$, so this did not happen. We end up with a large number of wavefronts building up in front of the object, forming a cone of high amplitude:
\begin{figure}[H]
    \centering
    \includegraphics[width=0.5\textwidth]{figures/lec09-04.png}
     \caption{A shock wave cone.}
\end{figure}

If $u = v$, this creates a shock wave which is a straight line tangent to the motion.
\begin{figure}[H]
    \centering
    \begin{minipage}{0.45\textwidth}
        \centering
        \includegraphics[width=0.6\linewidth]{figures/lec09-05.png}
        \caption{}
    \end{minipage}\hfill
    \begin{minipage}{0.45\textwidth}
        \centering
        \includegraphics[width=\linewidth]{figures/lec09-06.png}
        \caption{}
    \end{minipage}
\end{figure}

The angle created from the horizontal to either the top or bottom of this cone is known as the Mach Cone Angle $\theta$, and is given by the Mach Number $u/v$ (where v is velocity of the sound, u is the velocity of the source):
\[
    \sin \theta = \frac{v}{u}
\]

When travelling along the path of the moving source, the sound waves produced later are heard first. This is the reverse of what we'd expect for a slower source. This is because, as the source moves faster than sound, the source gets closer to the receiver faster than the old sound waves can travel. the sound waves emitted at (for example) $t = 10s$ are therefore emitted much closer to the receiver and arrive before the sound emitted at (for example) $t = 5s$.

At the back of the supersonic source, the frequency is significantly shifted:
\[
    f_b = \frac{v}{v+u}f_s < \frac{1}{2}f_s
\]

This equation suggests that in front of the source, the frequency becomes negative. This is because this analysis does not apply in front of the source, as waves do not pass the cone. The exception to this is a sonic boom, which is the extremely high amplitude of the cone itself being heard, as passes a receiver.

\section{Optics Part One}
Now we've studied waves, we can formally apply this to light. Electromagnetic waves are oscillations of the electric ($E$) and magnetic ($B$) fields.

\begin{figure}[H]
    \centering
    \includegraphics[width=0.75\textwidth]{figures/lec09-07.png}
     \caption{}
\end{figure}

With the following properties:
\begin{itemize}
    \item Light needs no medium, it can travel in a vacuum.
    \item Light does not involve the oscillation of particles (ignoring weird QM stuff).
    \item Light is a transverse wave (with $f, \omega, k$, standing waves, doppler shift etc).
\end{itemize}

For electromagnetic waves, we have the wave equation:
\[
    \frac{\partial^2 E}{\partial x^2} = \epsilon_0 \mu_0 \frac{\partial^2 E}{\partial t^2}
\]
Where $\epsilon$ and $\mu$ are two constants which determine how well waves propagate through electric and magnetic fields respectively. The zero subscript indicates that we are talking about free space (i.e. in a vacuum).

We have the following for the speed of light:
\[
    c = \sqrt{\frac{1}{\epsilon_0 \mu_0}}
\]
Hence:
\[
    c^2 = \frac{1}{\epsilon_0 \mu_0} \implies 1/c^2 = \epsilon_0 \mu_0
\]

Therefore the wave equation derived previously (with $1/v^2$ in place of $\epsilon_0 \mu_0$) is still satisfied.

For materials with \emph{relative permittivity $\epsilon_r$} and \emph{relative permeability $\mu_r$} (permittivity, $\epsilon$ and permeability $\mu$ for a specific material relative to the free-space values), the speed of light is:
\[
    v = \frac{c}{\sqrt{\epsilon_r \mu_r}} = \frac{c}{n}
\]

Where $n = \sqrt{\epsilon_r \mu_r} > 1$. Light therefore travels slower in matter.
