% !TeX root = main.tex
\lecture{13}{Wed 12 Nov 2025 11:00}{Image Formation}

\section{Sign Rules}
We have the mirror equation:
\[
    \frac{1}{s} + \frac{1}{s^\prime} = \frac{1}{f}
\]

The numerical values of $s$ and $s^\prime$ give us the distance, but the position of the formed image is determined by their signs.

\section{Mirrors}
The image formed by a mirror is:
\begin{itemize}
    \item \emph{Virtual}, as we effectively see it appears as being behind the mirror, where there is no real light.
    \item \emph{Erect}, as there is no inversion (if you stand in front of a mirror, you see your body in the same top-to-bottom orientation as real life).
    \item \emph{Reversed}, as if you look in a mirror your body faces you, and you see your face, not your back.
\end{itemize}

\subsection{Angled Mirrors}
If we have a pair of hinged mirrors, changing the angle between them changes the way that the image is produced. Crucially, it changes how many images of the object we see:
\begin{figure}[H]
    \centering
    \includegraphics[width=0.75\textwidth]{figures/lec13-01.png}
     \caption{}
\end{figure}
