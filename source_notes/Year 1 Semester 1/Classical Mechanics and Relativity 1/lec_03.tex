% !TeX root = main.tex
\lecture{3}{Tue 07 Oct 2025 12:00}{Kinematics Introduction}
For kinematics, we'll treat all objects as points and disregard aspects like rotation/the physical size of the body etc.

Given some point, we can define its position as a function of time $\vec{r}(t)$, and velocity as the derivative wrt time of this:
\[
    \vec{v}(t) = \frac{d \vec{r}}{dt}
\]

And acceleration:
\[
    \vec{a}(t) = \frac{d \vec{v}}{dt} =\frac{d^2 \vec{r}}{dt^2}
\]

\subsection{Position from Unit Vectors}
We can define:
\[
    \vec{r}(t) = r_x(t) \hat{e_x} + r_y(t) \hat{e_y} + r_z(t) \hat{e_z}\\
    = \sum_{j=1}^{3} r_j(t) \hat{e}_j
\]

So:
\[
    \frac{d \vec{r}}{dt} = \frac{d}{dt} \left(\sum_{j=1}^{3} r_j \hat{e}_j\right)
\]
\[
    = \sum_j \frac{d}{dt}(r_j \hat{e}_j)
\]
\[
    = \sum_j \frac{dr_j}{dt} \hat{e}_j
\]
\[
    \vec{v} = \sum_j v_j \hat{e}_j
\]

And:
\[
    \vec{a} = \frac{d \vec{v}}{dt} = \sum_{j=1}^{3} a_j \hat{e}_j
\]

Note: Taking the derivative of a vector wrt time is looking at how the variable changes in some infinitesimal time. This can be a change in direction, and/or a change in magnitude. To differentiate a vector we can differentiate it component-wise.

\subsection{Cartesian and Polar}
Instead of representing a point as x and y components (in 2D), we can instead define it as a distance from the origin $r$ and the angle this distance line forms with the positive x-axis $\theta$.

Therefore (by basic right angle trig) $x = rcos \theta$, $y = rsin \theta$, and hence:

\[
    \vec{r} = r \cos \theta \hat{e}_x + r \sin \theta \hat{e}_y
\]

So:
\[
    \vec{u(t)} = \frac{d \vec{r}}{dt} = \frac{d}{dt} (r \cos \theta) \hat{e}_x + \frac{d}{dt} (r \sin \theta) \hat{e}_y
\]

\[
    = \left(\dot{r}\cos \theta + r (-\sin \theta) \dot{\theta}\right) \hat{e}_x + \left(\dot{r} \sin \theta + r (\cos \theta) \dot{\theta}\right) \hat{e}_y
\]
\[
    = \dot{r} \left(\cos \theta \hat{e}_x + \sin \theta \hat{e}_y\right) + r \dot{\theta} \left(-\sin \theta \hat{e}_x + \cos \theta \hat{e}_y\right)
\]


\subsection{Example}
Lets model a particle, in a single dimension moving with constant acceleration ($a_0$) along a line. What is x(t)?

\textbf{The introduction of $\tau$ here was generally poorly understood by the class at the time. Please see Lec 05 for a more thorough explanation}.

\[
    a = \text{constant} = a_0
\]
\[
    a = a_0 = \frac{dv}{dt}
\]

So we can simply integrate to get $v(t)$ and again to get $x(t)$.

What if $a$ is not constant? Consider $a(t) = k t^3$. We begin by redefining $a(t)$ as the following, where tau is a time constant representing one time unit. This could be one second, one year etc.
\[
    a(t) = \tau^3 k \left(\frac{t}{\tau}\right)^3
\]
\[
    \text{let } a_* \equiv k \tau^3
\]
\[
    a(t) \equiv a_* \left(\frac{t}{\tau}\right)^3
\]

\[
    \int \, dv = \int a_* \tau \left(\frac{t}{\tau}\right)^3 \, d \frac{t}{\tau}
\]

\[
    v = \frac{1}{4} a_* \tau \left(\frac{t}{\tau}\right)^4 + v_0
\]

\emph{I've removed the rest here, as the whole $\tau$ thing was confusing in this lecture. Again, please see Lec 05, which re-does this section in a better manner.}