% !TeX root = main.tex
\lecture{10}{Thu 30 Oct 2025 15:00}{Terminal Velocity}

\section{Connected Bodies}
Consider $N$ blocks on a frictionless surface, all connected by an ideal string:
\begin{figure}[H]
    \centering
    \includegraphics[width=0.75\textwidth]{figures/lec10-01.png}
     \caption{}
\end{figure}

We pull the final block, with some magnitude of force $F$. All bricks are identical and have the same mass $m$.

\emph{What is the net force on brick number $i$, where i is arbitrary (and $i \leq N$). For example, what is the net force on brick $23$, denoted $F_{23}$}

The total acceleration of the entire system considered together is:
\[
    a = \frac{F}{M} \qquad M = Nm \text{  (total system mass)}
\]

And for our choice of brick:
\[
    F_{23} = m_{23} a_{23}
\]

As all the strings are ideal, the acceleration is equal for every brick, and is equal to the total system acceleration, hence:
\[
    F_{23} = m \frac{F}{M} = m \frac{F}{Nm} = \frac{F}{N}
\]

\section{Air Drag and Terminal Velocity}
We have some body moving with velocity $v$ through some medium (air, water etc). This medium has a frictional force (the `drag force') which acts upon the object in opposition to the direction of motion.

For sufficiently low velocities, this force is proportional to speed. However, as velocity increases this no longer applies, and may increase with (for example $v^2$). In a simple case of proportionality:

\[
    F_d = kv
\]
\[
    \vv{F_d} = -k v \hat{\underline{v}}
\]

\subsection{Terminal Velocity}
Consider jumping \footnote{with a parachute!} off a very tall tower. The only two forces that act upon you are:
\begin{itemize}
    \item $F_w = mg$ acting downwards.
    \item $F_d$ acting in the opposite direction to motion (upwards)
\end{itemize}

At some point, we have:
\[
    F_d = F_w
\]
At this time, there is no resultant force, no acceleration and therefore you move at a constant velocity. This velocity, which we will denote $v_*$ is the ``terminal velocity'' such that $F = a = 0$
\[
    F_d + F_w = 0
\]
\[
    -k v_* + mg = 0 \implies k v_* = mg \implies v_* = \frac{mg}{k}
\]

We want to investigate the behaviour of $v$ as it approaches terminal velocity, so want to build expressions for $v(t)$ and $z(t)$, where $z$ is the vertical axis.

\[
    F = ma
\]
\[
    mg - kv = ma
\]
\[
    mg - kv = m \frac{dv}{dt}
\]
\[
    \frac{mg - kv}{mg - kv} = \frac{m}{mg - kv} \frac{dv}{dt}
\]
\[
1 = \frac{m}{mg - kv} \frac{dv}{dt}
\]
\[
    dt = \frac{m}{mg - kv} dv
\]

And integrating:
\[
    \int_{t_0}^{t} \, dt = \int_{v_0}^{v} \frac{m}{mg - kv} \, dv
\]

Since our jump starts from rest at $t_0$, we have $t_0 = v_0 = 0$
\[
    \int_{0}^{t} \, dt = \int_{0}^{v} \frac{m}{mg - kv} \, dv
\]
\[
    t = \frac{m}{mg} \int_{0}^{v} \frac{1}{1 - \frac{k}{mg}v} \, dv
\]
\[
    = \frac{1}{g} \int_{0}^{v} \frac{1}{1 - \left(\frac{v}{v_*}\right)} \, dv
\]

We then stop integrating wrt $v$ alone and make a substitution such that we are integrating wrt the fraction of terminal velocity achieved:
\[
    t = \frac{v_*}{g} \int_{0}^{v/v_*} \frac{1}{1-\frac{v}{v_*}} \, d \left(\frac{v}{v_*}\right)
\]
\[
    t = \frac{v_*}{g} \left[- \log(1 - x)\right]_0^{v/v_*}
\]
\[
    t = \frac{v_*}{g} \left[-\log \left(1 - \frac{v}{v_*} + \log(1)\right)\right]
\]
\[
    t = -\frac{v_*}{g} \log \left(1 - \frac{v}{v_*}\right)
\]
\[
    -g \frac{t}{v_*} = \log \left(1 - \frac{v}{v_*}\right)
\]

Let $\tau \equiv v_* / g$:
\[
    - \frac{t}{\tau} = \log \left(1 - \frac{v}{v_*}\right)
\]
\[
    e^{-t / \tau} = 1 - \frac{v}{v_*}
\]
Hence:
\[
    v(t) = v_* \left(1 - e^{- t / \tau}\right)
\]
