% !TeX root = main.tex
\lecture{15}{Tue 18 Nov 2025 12:00}{Conservation of Linear Momentum}
\section{Curved Path Exercise Last Lecture}
Last lecture, we had a setup like this:
\begin{figure}[H]
    \centering
    \includegraphics[width=\textwidth]{figures/lec14-01.png}
     \caption{}
\end{figure}

An issue arose then in how we treat the mass over the curved section. We assume today that the mass travels from $C \to B$ and we've already solved the part from $A \to B$.

At the base of the ramp, the object has some velocity $v_0$. Previously, we neglected centripetal force.

Since we neglected centripetal acceleration, last lecture we got to a solution which is only valid if we consider the particle to be in static equilibrium with no centripetal acceleration at every single point, i.e when $v$ is infinitesimal. However, a reasonable velocity is required for the object to get up the ramp, so we have a physically inconsistent solution. 

We have a centripetal force in the same direction as the normal force:
\[
    F_c = N - mg \cos \theta = m r \omega^2 = mr \dot{ \theta}^2
\]

This is going to change as the object climbs, as the relative magnitude of relevant forces changes.

In the tangential direction, we have two forces, and a tangential acceleration given by circular motion as $mr \ddot{\theta}$:
\[
    -(mg \sin \theta + F_\text{fric}) = m | a |
\]
\[
      -(mg \sin \theta + F_\text{fric}) = mr\ddot{\theta}
\]

And using $F_\text{fric} = \mu N$ and $N = mr \dot \theta^2 + mg \cos \theta$:
\[
    -(mg \sin \theta + \mu (m r \dot \theta^2 + mg \cos \theta)) = mr \ddot \theta
\]
We haven't yet done the integration technique to actually solve this, but this is the correct form at least of the equation now we aren't neglecting centripetal acceleration!

\section{Conservation of Linear Momentum}
If there is no external force acting on a system in equilibrium:
\[
    \vv{F} = 0 \implies \frac{d \vv{\phi}}{dt} = 0 \implies \phi = \text{const.}
\]

So linear momentum does not change for a system in equilibrium unless a force is externally supplied. Linear momentum, like energy, must be conserved.

\subsection{Collisions}
For this section, we assume $u \ll c$ and special relativity can be disregarded. We consider two \emph{main} types of ideal collision although a collision in real life is not ideal, so can be in-between.
\begin{itemize}
    \item Totally inelastic collisions.
    \begin{enumerate}
        \item The two bodies stick together after the collision and behave together as one.
        \item Linear momentum is conserved.
        \item Kinetic energy \emph{of the two bodies} is not conserved, as some is lost to friction/sound/etc.
    \end{enumerate}
    
    \item Totally elastic collisions.
    \begin{enumerate}
        \item The two objects perfectly rebound off each other with no loss of energy (in real life some energy is dissipated, but this is an ideal)
        \item Linear momentum is conserved.
        \item Kinetic energy of the two bodies is conserved.
    \end{enumerate}
\end{itemize}

For simplicity, we consider motion in 1D, with two masses $m_1$ and $m_2$. Before the hit, they have velocities $u_1$ and $u_2$. After the collision, they have velocities $v_1$ and $v_2$.

We assume the collision is totally inelastic. Both particles are travelling in the same direction, but $m_1$ (on the left) is travelling faster than $m_2$ on the right, so catches up and collides from behind. 

After the collision, they travel together with mass $m = m_1 + m_2$ and velocity $v = v_1 = v_2$.

The linear momentum in the system before is:
\[
    m_1 u_1 + m_2 u_2
\]

And after:
\[
    (m_1 + m_2)v
\]

So conserving:
\[
    m_1 u_1 + m_2 u_2 = (m_1 + m_2)v
\]
\[
    \implies v = \frac{m_1 u_1 + m_2 u_2}{m_1 + m_2}
\]


We now assume that the collision is totally elastic, so we can conserve kinetic energy and we can conserve linear momentum.

\[
    m_1 u_1 + m_2 u_2 = m_1 v_1 + m_2 v_2
\]
\[
    \frac{1}{2}m_1 u_1^2 + \frac{1}{2} m_2 u_2^2 = \frac{1}{2}m_1 v_1^2 + \frac{1}{2}m_2 v_2^2
\]
\[
    \implies m_1 u_1^2 + m_2 u_2^2 = m_1 v_1^2 + m_2 v_2^2
\]

Therefore:
\[
    m_1 (u_1^2 - v_1^2) = m_2(v_2^2 - u_2^2)
\]
\[
    m_1 (u_1 + v_1)(u_1 - v_1) = m_2(v_2 - u_2)(v_2 + u_2) \tag{1}
\]

And as the masses don't change, conservation of momentum gives us that the sum of the velocities must be the same before and after. Rearranging the momentum equation:
\[
    m_1(u_1 - v_1) = m_2(v_2 - u_2) \tag{2}
\]

Dividing equation (1) by equation (2):
\[
    \frac{m_1 (u_1 + v_1)(u_1 - v_1)}{m_1(u_1 - v_1)} = \frac{m_2(v_2 - u_2)(v_2 + u_2)}{m_2(v_2 - u_2)}
\]
\[
    u_1 + v_1 = v_2 + u_2
\]

And the standard result:
\[
    v_2 - v_1 = -\ (u_1 - u_2)
\]