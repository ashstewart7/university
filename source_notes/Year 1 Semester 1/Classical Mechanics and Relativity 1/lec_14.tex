% !TeX root = main.tex
\lecture{14}{Thu 13 Nov 2025 15:00}{Conservation of Energy}

\section{Work Done Example I}
We have a straight portion of track with length $D$ which becomes a quarter circle ramp with radius $R$. We kick a box of ramp $m$, providing initial horizontal velocity $v_0$.

\begin{figure}[H]
    \centering
    \includegraphics[width=\textwidth]{figures/lec14-01.png}
     \caption{}
\end{figure}

\subsection{Assuming No Resistive Friction/Air Resistance}
The only forces on the block are gravity and the normal contact force. \emph{What $v_0$ must we impart on the block for it to (perfectly) reach the top of the ramp at point (B)?}.

Since there's no resistive forces along the horizontal section, the length of it is irrelevant. We can solve for the curved section using solely energy conservation:

\[
    \frac{1}{2}m v_0^2 = mgR + \xi
\]

We take the height of points (A) and (C) as the $U=0$ baseline, so $\xi = 0$.

\[
    \frac{1}{2} v_0^2 = gR
\]
\[
    v_0 = \sqrt{2gR}
\]

This was much easier compared to solving it using forces. Where possible, it's often extremely powerful to use conservation of energy in cases where we only care about the final and initial state of the object.

\subsection{Assuming Friction}
Assume there is some friction with $\mu_k = \mu$, we therefore have to account for work done by this:
\[
    E_A + W_{A \to B} = E_B
\]

Where $E$ is a general energy term encompassing both kinetic and potential.

\[
    W_{A \to B} = \int_{A}^{B} \vv{F}_\text{fric} \, d \vv{s}
\]

Breaking up the integral into the horizontal section and the ramp:
\[
    W_{A \to B} = \int_{A}^{C} \vv{F}_f \cdot d\vv{s} + \int_{C}^{B} \vv{F}_f \cdot d\vv{s}
\]
