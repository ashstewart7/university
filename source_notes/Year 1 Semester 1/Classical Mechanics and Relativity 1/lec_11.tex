% !TeX root = main.tex
\lecture{11}{Tue 04 Nov 2025 12:00}{Uniform Circular Motion and Work Done I}

\section{Uniform Circular Motion}
Consider a circle of radius $r$. A particle sits on this circle, travelling around it with velocity $\vv{v}$ and a force $\vv{F}$ acting from the particle towards the centre of the circle.

This velocity is tangential to the circle, so is perpendicular to the force.

\begin{figure}[H]
    \centering
    \includegraphics[width=0.4\textwidth]{figures/lec11-01.png}
     \caption{}
\end{figure}

The particle travels in a circular path with constant radius and constant speed. Consider the particle after some infinitesimal angle change $d \theta$. The direction vector of the particle (relative to the centre of the circle) goes from $\vv{r_1}$ to $\vv{r_2}$, which have the same magnitude but different directions.

There is some tiny change $d \vv{x}$ in horizontal position. This is given by:
\[
    d \vv{r} = \vv{r} d \theta
\]

The particle goes from $\vv{v_1}$ to $\vv{v_2}$ (again, same magnitude with slightly different direction). We therefore have:
\[
    d \vv{v} = \vv{v} d \theta
\]

\subsection{Determining Acceleration}
The magnitude of the infinitesimal displacement $|d\vv{r}|$ is related to $r$ and $d\theta$ by:
\[
    |d\vv{r}| = r \, d\theta \implies d\theta = \frac{|d\vv{r}|}{r}
\]
The velocity vectors rotate through the same angle $d\theta$ so we also have:
\[
    |d\vv{v}| = v \, d\theta
\]
\[
    |d\vv{v}| = v \left( \frac{|d\vv{r}|}{r} \right) = \frac{v}{r} |d\vv{r}|
\]
\[
    a = \frac{|d\vv{v}|}{dt} = \frac{v}{r} \frac{|d\vv{r}|}{dt}
\]
Using $\frac{|d\vv{r}|}{dt} = v$:
\[
    a = \frac{v}{r} (v) = \frac{v^2}{r}
\]

\subsection{Angular Frequency}
We define a new (constant) quantity called ``angular frequency'', $\omega$:
\[
    \omega \equiv \frac{d \theta}{d t} = \dot{\theta}
\]

The particle has time period to complete a whole rotation ($\theta = 2 \pi$), $T$. Therefore:
\[
    T = \frac{2 \pi}{\omega} \qquad \omega = \frac{2 \pi}{T}
\]

Using:
\[
    \frac{dr}{d \theta} = r \frac{d \theta}{dt}
\]
We have:
\[
    v = \omega r
\]
Hence:
\[
    a = \frac{\omega^2 r^2}{r} = \omega^2 r
\]

\section{Unit Vectors}
We define the position vector using the radial unit vector $\hat{e}_r$:
\[
    \vv{r} = r \hat{e}_r
\]
To differentiate this, we first define the unit vectors in Cartesian coordinates to determine their time derivatives:
\[
    \hat{e}_r = \cos\theta \, \hat{\imath} + \sin\theta \, \hat{\jmath} \quad \text{and} \quad \hat{e}_\theta = -\sin\theta \, \hat{\imath} + \cos\theta \, \hat{\jmath}
\]
Differentiating $\hat{e}_r$ wrt time:
\[
    \dot{\hat{e}}_r = \frac{d}{dt}(\cos\theta \, \hat{\imath} + \sin\theta \, \hat{\jmath}) = (-\sin\theta \cdot \dot{\theta}) \hat{\imath} + (\cos\theta \cdot \dot{\theta}) \hat{\jmath} = \dot{\theta} \hat{e}_\theta
\]
Differentiating $\hat{e}_\theta$ wrt time:
\[
    \dot{\hat{e}}_\theta = \frac{d}{dt}(-\sin\theta \, \hat{\imath} + \cos\theta \, \hat{\jmath}) = (-\cos\theta \cdot \dot{\theta}) \hat{\imath} - (\sin\theta \cdot \dot{\theta}) \hat{\jmath} = -\dot{\theta} \hat{e}_r
\]

\subsection{Velocity}
Using the product rule on $\vv{r} = r \hat{e}_r$:
\[
    \vv{v} = \frac{d}{dt}(r \hat{e}_r) = \dot{r} \hat{e}_r + r \dot{\hat{e}}_r
\]
Substituting $\dot{\hat{e}}_r = \dot{\theta} \hat{e}_\theta$:
\[
    \vv{v} = \dot{r} \hat{e}_r + r \dot{\theta} \hat{e}_\theta
\]

\subsection{Acceleration}
Using the product rule on $\vv{v}$:
\[
    \vv{a} = \frac{d}{dt}(\dot{r} \hat{e}_r) + \frac{d}{dt}(r \dot{\theta} \hat{e}_\theta)
\]
Expanding individual terms:
\[
    \frac{d}{dt}(\dot{r} \hat{e}_r) = \ddot{r} \hat{e}_r + \dot{r} \dot{\hat{e}}_r = \ddot{r} \hat{e}_r + \dot{r} (\dot{\theta} \hat{e}_\theta)
\]
\[
    \frac{d}{dt}(r \dot{\theta} \hat{e}_\theta) = (\dot{r} \dot{\theta} + r \ddot{\theta}) \hat{e}_\theta + r \dot{\theta} \dot{\hat{e}}_\theta = (\dot{r} \dot{\theta} + r \ddot{\theta}) \hat{e}_\theta + r \dot{\theta} (-\dot{\theta} \hat{e}_r)
\]
Grouping the $\hat{e}_r$ and $\hat{e}_\theta$ components:
\[
    \vv{a} = (\ddot{r} - r \dot{\theta}^2) \hat{e}_r + (r \ddot{\theta} + 2 \dot{r} \dot{\theta}) \hat{e}_\theta
\]

For UCM, radius is constant and $ \dot{\theta} = \omega$:
\[
    \vv{a} = (r \omega^2) \hat{e}_r + \hat{e}_\theta = -r \omega^2 \hat{e}_r
\]
Using $v = r \omega$:
\[
    \vv{a} = -\frac{v^2}{r} \hat{e}_r
\]
\[
    a = \frac{-v^2}{r}
\]

\section{Work Done by a Force}
We have some force $\vv{F}$ that causes the particle to move by a infinitesimal displacement $\vv{ds}$. We define the infinitesimal work done by this force as:
\[
    dW = \vv{F} \cdot d\vv{s}
\]

If the force acts at some angle $\theta$ to the direction of displacement, we have:
\[
    dW \equiv | \vv{F} | | d \vv{s} | \cos \theta = F \times ds \times \cos \theta
\]

This has dimensions of:
\[
    [dW] = \frac{ML}{T^2} L = \frac{M L^2}{T^2} = M \left(\frac{L}{T}\right)^2
\]

We notice this has the same dimensions as $mv^2$ which therefore gives us dimensions of energy.

Say a force now displaces the object along a (not necessarily straight, so not necessarily constant $\vv{F}$) path from point $(1)$ to point $(2)$. The work done by this motion is given by the integral:
\[
    W = \int_{(1)}^{(2)} \vv{F} \cdot ds
\]

\subsection{Example}
We again have some force with magnitude $F$, producing an infinitesimal displacement $d \vv{s}$. Say the two are in the same direction as each other:
\[
    dW = \vv{F} \cdot d \vv{s} = Fds 
\]

And if they are orthogonal:
\[
    dW = \vv{F} \cdot d \vv{s}
\]
\[
    = F ds \cos \frac{\pi}{2}
\]

And has $\cos \pi / 2 = 0$, the the force does no work in this case (cannot cause the displacement).

Finally, if the force and the displacement are in opposite directions to each other:
\[
    dW = \vv{F} \cdot d \vv{s} = F ds \cos \pi = -F ds
\]

So signs are important to uphold conservation of energy.

\subsection{Example II}
You, at point (1) want to travel to your friend along a beach at point (2) immediately below you. There is some frictional force opposing your motion ($F_\text{0}$).

What is the work against friction to go from (1) to (2), assuming you walk in a straight line.

\[
    W = \int_{(1)}^{(2)} \vv{F} \cdot d \vv{s}
\]
\[
    = - F_0 \int_{(1)}^{(2)}  \, ds
\]
 \[
     -F_0 ((2) - (1)) = -F_0 L
 \]
 
Say you are a little drunk. You take the path (walking three sides of a square instead of a straight line downwards):
\begin{itemize}
    \item $l$m at an angle of $90^\circ$ from the correct direction to your friend.
    \item You realise you've gone wrong, and walk $L$m downwards.
    \item You walk back $l$m at angle $90^\circ$ to finally meet them.
\end{itemize}

This gives us:
\[
    W = -F_0 l + -F_0 L - F_0 l = -F_0(2l + L)
\]

So we need to have our force and our path well defined along the whole route to be able to accurately determine work, as different paths will yield different works.