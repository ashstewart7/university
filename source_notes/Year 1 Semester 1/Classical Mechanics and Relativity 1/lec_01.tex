\lecture{1}{Tue 30 Sep 2025 12:00}{Orders of Magnitude and Dimensional Analysis}
Given some equation, for example $E = mv^2$, we can decompose it into the basic physical quantities that make it up, for example in terms of Mass, Time and Length. We can denote the dimensions of some quantity by wrapping it in square brackets.

\[
    E = \frac{1}{2}mv^2
\]
\[
    [E] = M L^2 T^{-2}
\]

\textbf{Example}:
\[
    \text{Pressure} \equiv \frac{\text{Force}}{\text{Area}}
\]

Suppose we want to test whether pressure and linear momentum flux (amount of linear momentum per unit time, per unit surface) were equivalent quantities, we could do this using dimensional analysis:

\[
    [P] = \frac{[F]}{[A]}
\]
\[
    [P] = \frac{M \times LT^{-2}}{L^2}
\]
\[
    = M / LT^2
\]

And for linear momentum flux ($\Phi(p)$ where lowercase p is momentum):
\[
    \Phi(p) = \frac{[p]}{[A][\text{time}]}
\]
\[
    = \frac{MLT^{-1}}{L^2 T}
\]
\[
    = \frac{M}{LT^2}
\]

So yes, they seem to be (at least dimensionally) equivalent.

\subsection{Challenging the LHC}
We want to use orders of magnitude calculations to challenge the idea that the LHC is the ``Big Bang Machine''.

The LHC operates on the order of magnitude of approx $10$TeV. The age of the universe is approx 13.7Bn Years, or (in orders of magnitude) $10^{10}$yrs.

What time was the big bang? The Big Bang started the universe, but we can't really say it happened at $0s$, because that doesn't really make sense. What about 1sec? or 1ms? Well it's clearly less than both of those, so we want to find the smallest possible increment of time ``Plank Second'' and say it happened after one of them.