% !TeX root = main.tex
\lecture{15}{Fri 20 Feb 2026 11:56}{Differential Equations III}

Summary of last lecture:
\begin{itemize}
    \item ``Homogeneous'' Equations
    \[
        \frac{dy}{dx} = f \left(\frac{x}{y}\right) \to x \frac{dv}{dx} = f(v) - v \text{ where } y(x) = xv(x)
    \]

    \item Linear Equations
    \[
        \sum_{k=0}^{n} a_k(x) \frac{d^k y}{dx^k} = f(x) \to y(x) = \sum_{k=1}^{n} \alpha_k y_k(x) + y_{PI}(x)
    \]
    The solution is a sum of the complementary function (the general solution of the homogenous equation) and the particular integral (one solution of the inhomogeneous equation).
    
    \item First Order Linear Equations
    \[
        \frac{dy}{dx} = P(x)y = Q(x) \to \frac{d}{dx} \left[y(x) e^{\int P(x) dx}\right] = Q(x) e^{\int P(x) dx}
    \]
    
\end{itemize}

\section{Examples}
\subsection{Example I}
\[
    (1-x^2) \frac{dy}{dx} - xy = 1
\]

Rewriting in the standard form:
\[
    \frac{dy}{dx} - \frac{x}{1-x^2} y = \frac{1}{1-x^2}
\]

Hence:
\[
    P(x) = \frac{-x}{1-x^2}
\]
\[
    I(x) = \exp \left(- \int \frac{xdx}{1-x^2}\right) = \exp \left(\frac{1}{2} \ln(1-x^2)\right) = \sqrt{1 - x^2}
\]

Multiplying through by the integrating factor:
\[
    \sqrt{1 - x^2} \frac{dy}{dx} - \frac{x}{\sqrt{1 - x^2}} y = \frac{1}{\sqrt{1-x^2}}
\]

The L.H.S is now the derivative of a product:
\[
    \frac{d}{dx} \left(y \sqrt{1-x^2}\right) = \frac{1}{\sqrt{1-x^2}}
\]
And integrating both sides:
\[
    y \sqrt{1-x^2} = \arcsin x + c
\]
\[
    \implies \boxed{y = \frac{c}{\sqrt{1 - x^2}} + \frac{\arcsin x}{\sqrt{1-x^2}}}
\]

\subsection{Example II}
%TODO  

\section{Linear ODEs with Constant Coefficients}
In the most general form, we have:
\[
    a_n \frac{d^n y}{dx^n} + a_{n-1} \frac{d^{n-1}y}{dx^{n-1}} + \cdots + a_1 \frac{dy}{dx} + a_0y = f(x)
\]

Firstly, we solve the homogenous equation:
\[
    a_n \frac{d^n y}{dx^n} + a_{n-1} \frac{d^{n-1}y}{dx^{n-1}} + \cdots a_1 \frac{dy}{dx} + a_0y = 0
\]

Let $y(x) = e^{\lambda x}$:
\[
    e^{\lambda x} \left(a_n \lambda^n + a_{n-1} \lambda^{n-1} + \cdots + a_1 \lambda + a_0\right) = 0
\]

This reduces to finding the zeroes of the corresponding nth order polynomial. If this polynomial has $n$ distinct zeroes, then the complimentary function is given by:
\[
    y_{CF}(x) = \alpha_1 e^{\lambda_1x} +  \alpha_2 e^{\lambda_2x} + \cdots +  \alpha_n e^{\lambda_nx}
\]

If these roots contain a repeated root, this will reduce the number of unique solutions by one. If we have any complex solutions, they wil come in complex conjugate pairs:
\[
    a^{( \pm i b)x} = e^{ax} \left(\cos bx \pm i \sin bx\right)
\]

This therefore has independent real solutions:
\[
    e^{ax} \cos bx
\]
\[
    e^{ax} \sin bx
\]

\section{Equidimensional Equations}
These have coefficients which do depend on $x$, but where the coefficients are functions of $x$ such that the $n$th derivative has a coefficient of $a_n x^n$

The general form of a homogeneous equidimensional equation is:
\[
    a_n x^n \frac{d^n y}{dx^n} + a_{n-1} x^{n-1} \frac{d^{n-1}y}{dx^{n-1}} + \cdots + a_1 x \frac{dy}{dx} + a_0y = f(x)
\]

This has a solution in the general form $y(x) = x^\lambda$. When we differentiate $k$ times with respect to $x$ we lose $k$ powers of $x$, as each differentiation decreases the power. We need to restore this therefore with a $x^k$ prefactor.

\section{Mass on a Spring (SHM)}
A mass, $m$ on a spring is displaced from its equilibrium position by some distance $x$. There is a restoring force given by $F = -kx$.

\[
    F = ma \implies F = m \frac{d^2}{dt}
\]
\[
    m \frac{d^2 x}{dt^2} + kx = 0
\]
\[
    \frac{d^2 x}{dt^2} + \omega^2 X = 0 \qquad \text{where } \omega^2 = \frac{k}{m}
\]

We have:
\[
    x(t) = e^{\lambda t}
\]
\[
    (\lambda^2 + \omega^2) e^{\lambda t} = 0
\]
\[
    \lambda^2 + \omega^2 = 0
\]

Hence:
\[
    \lambda = e^{\pm i \omega t}
\]




