% !TeX root = main.tex
\lecture{12}{Fri 13 Feb 2026 12:00}{End of Partial Differentiation \& Start of ODEs}

Recap of lecture 11:

\begin{itemize}
    \item The tangent plane to a surface $f(x, y, z) = 0$ at $(x_0, y_0, z_0)$ is given by:
    
    \[
        \left(\frac{\partial f}{\partial x}\right)_0 (x - x_0) + \left(\frac{\partial f}{\partial y}\right)_0 (y - y_0) + \left(\frac{\partial f}{\partial z}\right)_0 (z - z_0) = 0  
    \]

    Such that $\underline{\nabla}f(x_0, y_0, z_0)$ is the normal vector to the plane

    \item The parametric representation of a curve $\underline{r}(t)$ has:
    \begin{itemize}
        \item Unit tangent: $\underline{\hat{T}} = \frac{d \underline{r}}{dt} / \left|\frac{d \underline{r}}{dt}\right|$
        \item Arc length $s(t)$: $\frac{ds}{dt} = \left|\frac{d \underline{r}}{dt}\right| \to \hat{\underline{T}} = \frac{d \underline{r}}{ds}$.
        \item Unit normal and curvature:
    \end{itemize}
\end{itemize}

\section{Orthonormal Triads}
We can create an \emph{orthonormal triad} by introducing a new normal vector called the unit binormal, $\hat{\underline{B}} = \hat{\underline{T}} \times \hat{\underline{N}}$.

Since $\hat{\underline{N}} \times \hat{\underline{N}}$, differentiating wrt $s$ gives:
\[
    \hat{\underline{N}} \cdot \frac{d\hat{\underline{N}}}{ds} = 0
\]

TODO

We have:
\[
    \frac{d \hat{\underline{T}}}{ds} = \kappa \hat{\underline{N}}
\]
\[
    \frac{d \hat{\underline{N}}}{ds} = - \kappa \hat{\underline{T}} + \tau \hat{\underline{B}}
\]

Hence:
\[
    \frac{d \hat{\underline{B}}}{ds} = \frac{d}{ds} \left(\hat{\underline{T} \times \hat{\underline{N}}}\right)
\]
\[
    = \frac{d \hat{\underline{T}}}{ds} \times \hat{\underline{N}} + \hat{\underline{T}} \times \frac{d \hat{\underline{N}}}{ds}
\]
\[
    = \kappa \hat{\underline{N}} \times \hat{\underline{N}} + \hat{\underline{T}} \times \left(- \kappa \hat{\underline{T}} + \tau \hat{\underline{B}}\right)
\]
\[
    = \tau \hat{\underline{T}} \times \hat{\underline{B}} = \tau \hat{\underline{T}} \times \left(\hat{\underline{T}} \times \hat{\underline{N}}\right)
\]
\[
    = \tau\left[\left(\hat{\underline{T}} \cdot \hat{\underline{N}}\right) \hat{\underline{T}} - \left(\hat{\underline{T}} \cdot \hat{\underline{T}}\right) \hat{\underline{N}}\right]
\]

\[
    = \tau \hat{\underline{N}}
\]

This gives the Frenet-Serret Formulae:

\textbf{This concludes partial differentiation! :D}

\section{Ordinary Differential Equations}
A differential equation is any equation that involves derivatives. We care, because most laws of physics manifest themselves in the form of differential equations. For example:

\[
 \text{Newton's Second Law: } \qquad \underline{F} = m \frac{d^2 \underline{r}}{dt^2}
\]

\[
    \text{3D Time-Independent Schrödinger Eqn: } \qquad - \frac{\hbar}{2m} \left(\frac{\partial^2 \psi}{d x^2} + \frac{\partial^2 \psi}{d y^2} + \frac{\partial^2 \psi}{d z^2}\right) + V(x, y, z) \psi = E \psi
\]

\[
    \text{3D Wave Eqn: } \qquad \frac{\partial^2 \phi}{\partial x^2} + \frac{\partial^2 \phi}{\partial y^2} + \frac{\partial^2 \phi}{\partial z^2} = \frac{1}{c^2} \frac{\partial^2 \phi}{\partial t^2}
\]
\[
    = \nabla^2 \phi = \frac{1}{c^2} \frac{\partial^2 \phi}{\partial t^2}
\]

\[
    \text{Gauss' Law: } \qquad \frac{\partial E_x}{\partial x} + \frac{\partial E_y}{\partial y} + \frac{\partial E_z}{\partial z} = \frac{\rho}{\epsilon_0}
\]

\[
    \text{Navier-Stokes Eqn: } \qquad \rho \left(\frac{\partial \underline{v}}{\partial t} + \left(\underline{v} \cdot \underline{\nabla}\right) \underline{v}\right) = - \underline{\nabla}p + \rho \underline{g} + \mu \nabla^2 \underline{v} 
\]

In this course, we will only solve DEs of a single variable, i.e. Ordinary Differential Equations (ODEs). We don't look at Partial DEs of multiple variables yet.

In order to think about solving these, we need to classify them. Most DEs aren't soluble in closed form with elementary functions and need to be solved numerically. Here, we only consider nice soluble functions, but this is a vast minority in reality. We want to identify classes of DEs we can reasonable solve with a method for each.

We can generally solve linear equations by breaking them into small chunks and solving them individually, for example.

\section{Types of DEs}
\subsection{Partial vs. Ordinary}

In the examples above, only the first was an ODE, and the rest PDEs. Ordinary Differential Equations (ODEs) involve only a single variable. 

Consider a vector $\underline{r}(t) = (x(t), y(t), z(t))$. $t$ is called the independent variable, with $x, y, z$ being dependant variables. While we have 3 dependant variables, we only have one independent variable (so only one thing to differentiate wrt), so this would end up being ordinary.

PDEs involve equations of two or more variables and hence involve partial derivatives.

\subsection{Order}
The order of a DE is given by the order of the highest derivative involved, so Newton's 2nd Law is a second order DE, as the highest order derivative is a second derivative.

\subsection{Degree}
The degree of a DE is a less important measure than the others. It is given by the highest power of the highest order derivative. For example, Newton's 2nd is a first order, while an equation containing $a^3$ would be third degree (and second order, as $a$ is a second derivative).

Ideally, we want this to be $1$ for ease of solving, and higher degrees are rare but they do exist. For example, from Lagrangian Mechanics we have:
\[
    \frac{1}{2m} \left[\left(\frac{\partial s}{\partial x}\right)^2 + \left(\frac{\partial s}{\partial y}\right)^2 + \left(\frac{\partial s}{\partial z}\right)^2\right] + V(x, y, z) = \frac{ds}{dt}
\]

\subsection{Homogenous and Inhomogeneous}
A homogenous DE is a DE that does not have any terms of only the independent variable(s), while an inhomogeneous DE does.

For example, Newton's 2nd is homogenous as there is no term that involves $t$ alone. This would be inhomogeneous:

\[
    \frac{\partial^2 x}{\partial t^2} = t + x
\]

While this would be homogenous:
\[
    \frac{\partial^2 x}{\partial t^2} = tx
\]

As $t$ is a coefficient and not a pure term in its own right.

\subsection{Linear and Non-Linear}
A DE is linear if the dependant variable(s) and all of its/their derivatives occur purely as linear functions. For example:
\[
    \left(1 - x^2\right) \frac{d^2 y}{dx^2} - 2x \frac{dy}{dx} + n(n+1)y = 0
\]

This is linear, as the dependant variable $y$ never has a coefficient greater than $1$.

\[
    \frac{dy}{dx} + xy = 0
\]

Is also linear, while this is not:

\[
    \frac{dy}{dx} + xy^2 = 0
\]

This is also non-linear (as shown by the Taylor Expansion of sine):
\[
    \frac{d^2 \theta}{dt^2} = - \frac{g}{l} \sin \theta
\]

\subsection{Examples}
\[
    (1) \quad \left(\frac{\partial u}{\partial x}\right)^2 + \left(\frac{\partial u}{\partial y}\right)^2 = u^2
\]
Homogenous first-order second-degree non-linear PDE.

\[
    (2) \quad \frac{\partial^2 u}{\partial x^2} + \frac{\partial^2 u}{\partial y^2} + \frac{\partial^2 u}{\partial z^2} = x^2 + y^2 + z^2
\]
Inhomogeneous second-order first-degree linear PDE.

\[
    (3) \quad \frac{\partial y}{\partial x} + y^2 = x
\]
Inhomogeneous first-order second-degree non-linear ODE.





