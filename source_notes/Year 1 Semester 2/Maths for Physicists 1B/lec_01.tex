% !TeX root = main.tex
\lecture{1}{Wed 21 Jan 2026 11:00}{Course Welcome and Introduction to Partial Differentiation}

\section{Course Welcome}
\subsection{Recommended books:}
\begin{itemize}
    \item Mathematical Techniques 4e, Jordan \& Smith
    \item Engineering Mathematics 8e, Stroud
    \item Calculus (Schaum), 6e, Ayres \& Mendelson
    \item Advanced Calculus (Schaum), 6e, Ayres \& Mendelson
\end{itemize}

\subsection{Assessment details:}
\begin{itemize}
    \item Maths 1A/1B form a single 20 credit module.
    \item 80\% assessed by a 3 hour exam - Section 1 is 36\% with 6 short questions and Section 2 is 64\% with 4 long questions.
    \item 20\% assessed by problem sheets.
\end{itemize}

\subsection{Course structure:}
\begin{enumerate}
    \item Partial Differentiation
    \begin{itemize}
        \item Definition, total differential, chain rule, gradient.
        \item Taylor series, stationary points, Lagrange multipliers.
    \end{itemize}
    \item Differential Equations
    \begin{itemize}
        \item Definition, 1st order separable, exact and homogenous.
        \item Linear equations: general solution, 1st order and constant coefficients.
    \end{itemize}
    \item Integration
    \begin{itemize}
        \item Definition as area under the curve, fundamental theorem of calculus.
        \item Integration by: substitution, parts, partial fractions and tricks.
    \end{itemize}
    \item Multiple Integrals
    \begin{itemize}
        \item Multiple and repeated integrals. Change of order of integration.
        \item Change of variables and the Jacobian. Arc length. Solids of revolution.
    \end{itemize}
\end{enumerate}

\section{Multivariate Functions}
Lots of physics involves functions of more than one variable. A physical quantity defined at every point in space is called a field. We can have both scalar fields and vector fields.

For example, some scalar fields are:
\begin{itemize}
    \item $V(x, y, z)$: Electrostatic potential. This is often easier to work with compared to the full electric (vector) field.
    \item $T(x, y, z)$: Temperature.
    \item $p(x, y, z)$: Pressure.
\end{itemize}

While some vector fields are:
\begin{itemize}
    \item $\underline{E}(x, y, z)$: Electric Field.
    \item $\underline{B}(x, y, z)$: Magnetic Field.
    \item $\underline{v}(x, y, z)$: Velocity Field (i.e in fluid mechanics).
\end{itemize}

\subsection{Partial Derivatives}
Consider a function of two variables. The partial derivative of a function with respect to one variable is the rate of change of a function wrt that variable, while keeping other variables constant. Effectively, we carry out a derivative while treating the other variables as if they were constants.

Suppose we have a function $f(x, y)$. The definition of a partial derivative is:
\[
    \frac{\partial f}{\partial x}(x_0, y_0) = \lim_{h \to 0} \frac{f((x_0 + h), y_0) - f(x_0, y_0)}{h}
\]
\[
    \frac{\partial f}{\partial y}(x_0, y_0) = \lim_{k \to 0} \frac{f(x_0, (y_0 + k)) - f(x_0, y_0)}{k}
\]

Just like we denote $\frac{df}{dx}$ as the derivative of a function of a single variable, we denote $\frac{\partial f}{\partial x}$ as the partial derivative of a function of several variables.

Note that this is not delta f and delta y, i.e. not $\frac{\delta f}{\delta x}$

In theory, we'd explicitly notate:
\[
    \left(\frac{\partial f}{\partial x}\right)_y
\]
With the subscript $y$ explicitly stating that $y$ is being kept constant. This is rarely, but sometimes, needed.

Consider $f(x, y, z) = x^2 \sin yz$. We have:
\[
    \frac{\partial f}{\partial x} = 2x \sin yz
\]
\[
    \frac{\partial f}{\partial y} = x^2 z \cos yz
\]
\[
    \frac{\partial f}{\partial z} = x^2 y \cos yz
\]

\subsection{Higher Orders}
Higher derivatives are defined as they were previously, but they can now be mixed. For example, with $f(x, y) = x^2 \sin y$, we can write:
\[
    \frac{\partial f}{\partial x} = 2x \sin y \qquad \frac{\partial f}{\partial y} = x^2 \cos y
\]
\[
    \frac{\partial^2 f}{\partial x^2} = 2 \sin y
\]

We can also have:
\[
    \frac{\partial^2 f}{\partial y \partial x} = \frac{\partial}{\partial y} \left(\frac{\partial f}{\partial x}\right) = 2x \cos y
\]

Shorthand notation exists, i.e. $f_{xx}$ = $\frac{\partial^2 f}{\partial x^2}$ or $f_{yx} = \frac{\partial^2 f}{\partial y \partial x}$

For most cases, but not all, mixed derivatives are often independent of the order of partial derivatives, so: $f_{xy} = f_{yx}$