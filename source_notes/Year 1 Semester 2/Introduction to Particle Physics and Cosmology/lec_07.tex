% !TeX root = main.tex
\lecture{1}{Mon 23 Feb 2026 12:00}{Start of Cosmology: A Brief Introduction to Cosmology}

\paragraph{Primary Textbook:} ``An Introduction to Modern Cosmology'', Andrew Liddle. Each lecture will have a chapter or two as recommended reading from this book and is available as an ePDF from the library.

\section{The origin of Cosmology}
Comes from the Ancient Greek word $\kappa o \sigma \mu o \sigma$, ``kosmos'' meaning order. Cosmology is a branch of astronomy concerned with studying the origin and evolution of the universe. We won't really consider anything on a smaller scale than clusters of galaxies.

The earliest form of this is \emph{celestial mechanics} which goes back to Ancient Greek philosophers sch as Aristotle and Ptolemy.

\subsection{Geocentric Ptolemaic System}
This was the prevailing theory until the 16th century where the earth was at the centre of the universe and other stars/planet etc revolved around the earth in ``epicycles''. Here, the planets sat on rotating spheres, which orbited the earth.

\begin{figure}[H]
    \centering
    \includegraphics[width=0.75\textwidth]{figures/lec07-01.png}
     \caption{}
\end{figure}

The epicycles explain retrograde motion. Observational power was too limited to get more accurate measurements, and they noticed that some objects appeared to go backwards as they were moving forwards overall in the orbit. We can solve this by placing another smaller circular orbit superimposed on the larger orbit. 

We now know that this is not accurate, but it explained the motion at the time.

\subsection{Heliocentric System}
Copernicus in 1543 radically changed the understanding of orbits. He suggested that the sun, not the earth, was near the centre of the universe. The planets, including the earth revolved in circles around the sun, with other stars orbiting much further away.

In order to match the observed motion, we still need epicycles in this initial version of the theory.

This was the foundation of modern astronomy, and lead to:
\begin{itemize}
    \item Kepler discovering elliptical orbits. This got rid of the need for epicentres and described the motion of planets with the data collected at the time.
    \item Galilei discovering the moons of Jupiter and phases of Venus. Before Galilei, these telescopes were not sufficiently powerful to observe these phases.
    \item Newton's theory of universal gravity in 1678.
\end{itemize}

This theory, and the implications it had on humanity's role in the universe was deeply unpopular with religious leaders and this caused opposition to its widespread acceptance.

This was a better way to think about the solar system (which was about the limit of what was practically observable). Copernicus therefore needed to provide evidence to justify this. These later from Kepler, Galilei and Newton acted as this evidence.

\subsection{From Herschel to Hubble}
In order to go from scales on a solar system level to a universal level, we need more powerful measurement equipment to be able to gather as much data as possible. William and Caroline Herschel pioneered this by creating the most powerful telescopes to date in 1877.

They probed deeper and deeper past the solar system, in an attempt to work out whether or not there was some kind of edge to the universe. They discovered diffuse objects (``nebulae'') beyond the milky way, and (as did Charles Messier) started to build up a catalogue of objects in the universe.

\subsection{Shapley-Curtis Debate}
%TODO





