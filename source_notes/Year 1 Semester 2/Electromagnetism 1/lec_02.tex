% !TeX root = main.tex
\lecture{12}{Thu 26 Feb 2026 11:00}{Currents and Magnetic Force}

Today we start Part II, Magnetism:
\begin{itemize}
    \item Definition of current and current density.
    \item Magnetic force on a moving charge.
    \item The Lorentz Force.
    \item Field lines for a magnetic field.
\end{itemize}

\section{Current}
\subsection{Definition of Current}
Suppose a current carries a current $I$, this is defined in terms of the rate of flow of charge past a given cross section:
\[
    I = \frac{dQ}{dt}
\]

The SI unit of current is the \emph{ampere}, defined as the flow of one coulomb per second.

\subsection{Nature of Current}
There's two experimentally observed effects of a current flow:
\begin{itemize}
    \item Heating
    \item Creation of magnetic fields.
\end{itemize}

Consider a current through a conductor. This is caused by electrons moving (with average drift velocity $v$) as a result of the induced E-field.

In some time, they travel $v \Delta t$ through the cross section, and if this cross section has area $A$ they sweep out an area of $A v \Delta t$.

If a unit volume has $n$ conducting free electrons, the number of charges is given by:
\[
    N = n(A v \Delta t)
\]

And hence the total charge is:
\[
    \Delta Q = nAv \Delta T (-e)
\]

Hence:
\[
    \hat{\imath} = \frac{\Delta Q}{\Delta t} = -nA e \underline{v}
\]

\subsection{Current Density}
More usefully, we talk about the current per unit cross-section, which is the current density $J$:
\[
    \hat{\jmath} = \frac{\hat{\imath}}{\underline{A}} = -ne \underline{v}
\]

The negative sign arises as we define current as flowing from positive to negative. Electrons, being negatively charged, actually physically flow from negative to positive. Current flow and actual electron flow are therefore in the opposite direction. This is because current was initially understood and defined before the electron was defined.

Typically, $v < 1 \si{mm s^{-1}}$ when a current is flowing.

\section{Magnetism}
\subsection{Magnetic Force on a Charge}
A magnetic field exerts a force on a moving charge that is present in the magnetic field.

Suppose a particle of charge $+q$ moving with some velocity $\underline{v}$ in a magnetic field $\underline{B}$ experiences some force $\underline{F}_m$.

We cannot derive from first principles, but experimentally have observed:
\begin{itemize}
    \item $\underline{F}_m $ $\perp \underline{v}, \underline{B}$.
    \item $\underline{F}_m \propto \underline{v}$.
    \item $\underline{F}_m \propto \underline{B}$.
    \item $\underline{F}_m \propto q$
\end{itemize}

If $\theta$ is the angle between the velocity and the B-field direction, we hae:
\[
    \boxed{\underline{F}_m = Bqv \sin \theta}
\]

In vector form:
\[
    \boxed{\underline{F}_m = q \underline{v} \wedge \underline{B}}
\]

Here, $\underline{B}$ has the unit Tesla, T. In base units this is $\si{N C^{-1} m^{-1} s}$. A one Tesla magnetic field is really quite powerful, so we also define the Gauss, $1 \si{G} = 10^{-4} \si{T}$ to move to a nicer scale.

By definition, if a $1$C charge moving at $1$m/s perpendicular to a magnetic field experiences a force $1$N, the field is $1$T.

\begin{itemize}
    \item Earths magnetic field is $\sim 0.5$G
    \item Poles of a large electromagnetic: $\sim 2$T.
    \item Surface of a neutron star: $\sim 10^8$T.
    \item The maximum pulsed magnetic field that can create in a lab: $\sim 50$T.
\end{itemize}

The Earth's field is believed to be generated by electron currents in the iron alloys in its core. It's not completely understood yet, but convection currents causing these conductive alloys to flow and move is the working theory. This movement causes the North and South poles to swap places on average every $300,000$ years.

\subsection{The Lorentz Force}
In a region where we have both an E-field and a B-field, we have the total force as the vector sum of both:
\[
    \underline{F} = q \left(\underline{E} + \underline{v} \wedge B\right)
\]

This is called the Lorentz force.

Suppose Earth's magnetic field is given by:
\[
    \underline{B} = B \cos 70^\circ \hat{\jmath} - B \sin 70^\circ \hat{k} \qquad \text{where: } B = 5 \times 10^{-5} \si{T}
\]

A proton moves in this field with:
\[
    \underline{v} = 10^7 \hat{\jmath} \si{ms^{-1}}
\]

\[
    \underline{F}_m = +e \underline{v} \wedge \underline{B} = 1.6\times 10^{19} \times 10^7 \times \det\begin{bmatrix}
    \hat{\imath} & \hat{\jmath} & \hat{k} \\
    0 & 1 & 0 \\
    0 & B_y & B_z \\
    \end{bmatrix}
\]
\[
    = 1.6 \times 10^{-12} \left[\hat{\imath} \left(B_z - 0\right) + \hat{\jmath} \left(0-0\right) + \hat{k} \left(0-0\right)\right]
\]
\[
    = 1.6 \times 10^{-12} B_Z \hat{\imath}
\]
\[
    = - 1.6 \times 10^{-12} \times 5 \times 10^{-5} \sin 70^\circ \hat{\imath}
\]
\[
    - 7.5 \times 10^{-17} \hat{\imath} \si{N}
\]


\subsection{Magnetic Field Lines}
\begin{figure}[H]
    \centering
    \includegraphics[width=0.6\textwidth]{figures/lec12-01.png}
     \caption{Magnetic Field Lines Around a Bar Magnetic}
\end{figure}

Magnetic field lines are different to electrical field lines. They do not point in the direction of force and travel out of North poles and into South poles. Magnetic field lines are always continuous so there is no magnetic monopoles.

Physicists have searched for a potential magnetic monopoles, and the current upper limit for the number of magnetic monopoles per body is $<10^{-29}$.

The tangent to a field line at some point gives the direction of $B$ at that point. The number of field lines drawn per unit cross section $\propto$ B.

\subsection{Magnetic Flux}
The magnetic flux $\Delta \phi_B$ has the same definition as electric flux in an E-field. For a small area $\Delta A$:
\[
    \Delta \phi_B = B \cos \theta \Delta A
\]

Where $\theta$ is the angle between the B-field and the surface normal vector.

The total magnetic flux is therefore the integral of this:
\[
    \phi_B = \int_S \underline{B} \cdot d \underline{S}
\]

As there is no monopoles, and exiting flux must also return. Therefore, the net flux over any enclosed surface is zero:
\[
    \int_S \underline{B} \cdot d \underline{S} = 0
\]

While this does form one of Maxwell's equations and is very useful for EMII next year, it's not useful in EMI\dots













