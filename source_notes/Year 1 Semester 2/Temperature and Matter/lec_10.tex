% !TeX root = main.tex
\lecture{10}{Thu 19 Feb 2026 13:00}{Thermodynamic Transitions}

\section{Thermodynamic Transitions}
\subsection{Isothermal (Fixed Temperature)}
This may look like:
\begin{itemize}
    \item A curve on a PV plot. As $PV = nRT$, a PV plot shows a $1/x$ relationship for constant $T$.
    \item A vertical straight line on a PT plot.
    \item A vertical straight line on a VT plot.
\end{itemize}

If the temperature change is $0$, $\Delta U$ (which we derived last time only depends on temperature) follows:
\[
    \Delta T = \Delta U = 0
\]

Hence:
\[
    0 = Q_\text{in} + W_\text{on}
\]

\subsection{Isochoric (Fixed Volume)}
This may look like:
\begin{itemize}
    \item A vertical straight line on a PV plot.
    \item A proportional relationship on a PT plot.
    \item A horizontal straight line on a VT plot.
\end{itemize}

For an isochoric transition, $\Delta U = Q_in$, as:
\[
    W_\text{on} = (-) \int P dV
\]

And as $dV = 0$, $W_\text{on} = 0$

\subsection{Isobaric (Fixed Pressure)}
This may look like:
\begin{itemize}
    \item A horizontal straight line on a PV plot.
    \item A horizontal straight line on a PT plot.
    \item A proportional relationship on a VT plot.
\end{itemize}

For an isobaric transition, $P$ is just a constant, so the integral:
\[
    W_\text{on} = \int_{V_1}^{V_2} P \, dV = \left[PV\right]_{v_1}^{v_2}
\]

Where $P$ is constant, which is nice and easy. Generally:
\[
    W_\text{on} = - \int P dV = (-) P \Delta V
\]

\[
    \Delta U = \Delta Q - P \Delta V
\]


\section{Heat Capacities of Ideal Gases}
Recall that:
\[
    \Delta Q = C \Delta T \implies C = \frac{dQ}{dT}
\]

At a fixed volume:
\[
    C_V = \left(\frac{dQ_\text{in}}{dT}\right)_T
\]

And at a fixed pressure:
\[
    C_P = \left(\frac{dQ_\text{in}}{dT}\right)_P
\]

Mayer's Relation says that:
\[
    C_P = C_V + nR
\]

This is an examinable derivation.

\subsection{Deriving Mayer's Relation}
\begin{proof}
Consider a gas piston. We have state variables $V, P, T$, and $N$ is constant as the piston is sealed. The first law of thermodynamics says:
\[
    \Delta U = Q_\text{in} + W_\text{on}
\]

We can consider this for both a constant volume or a constant pressure. For a constant volume:
\[
    \Delta U = \Delta Q_1
\]

And for a constant pressure:
\[
    \Delta U = \Delta Q_2 - P \Delta V
\]

Noting that the two $Q$s are different as we have two different transitions. We can create a slightly different expression for constant volume by multiplying by $\Delta T / \Delta T$:
\[
    \Delta U = \frac{\Delta Q_1}{\Delta T} \times \Delta_T = C_V \Delta T
\]

And for constant pressure:
\[
    \Delta Q_2 = \Delta U + P \Delta V = \frac{\Delta Q_2}{\Delta T} \Delta T = C_P \Delta T
\]

Hence:
\[
    C_P \Delta T = \Delta U + P \Delta V
\]
\[
    C_V \Delta T = \Delta U
\]

Setting equal to each other based on $\Delta U$:
\[
    \Delta U = \boxed{C_V \Delta T = C_P \Delta T - P \Delta V}
\]
\[
    C_P \Delta T - C_V \Delta T = P \Delta V
\]
\[
    \Delta T \left(C_P - C_V\right) = P \Delta V
\]
\[
    \implies C_P - C_v = \frac{\Delta V}{\Delta T} \times P \qquad \text{NB: $P$ is still constant.}
\]

Using $PV = nRT$, for constant pressure we have $\frac{P \Delta V}{\Delta T} = nR$. Hence:
\[
    C_P - C_V = nR
\]
\[
    C_P = C_V + nR
\]
\end{proof}

It is clear that $C_P > C_V$.

Note: We tend to use heat capacity at a constant volume unless specified.


\section{Back to Thermodynamic Transitions}
\subsection{Adiabatic Transitions (No Heat Transfer)}
These seem quite a bit like isothermal transitions, but look different on a PV plot:
\begin{figure}[H]
    \centering
    \includegraphics[width=0.75\textwidth]{figures/lec10-01.png}
     \caption{Differences in transitions on a PV Plot}
\end{figure}

\begin{figure}[H]
    \centering
    \includegraphics[width=0.75\textwidth]{figures/lec10-02.png}
     \caption{Adiabatic Transitions on a PV and VT Plot}
\end{figure}

For isothermal transitions, $PV = \text{const}$.

For adiabatic transitions, $PV^\gamma = \text{const}$ where $\gamma = \frac{C_P}{C_V}$

In a PV plane, adiabatic processes appear steeper than isothermal processes and adiabatic processing are steeper in the PV plane than the VT plane.

For an adiabatic transition, we can reduce the first law of thermodynamics like so:
\[
    \Delta U = Q_\text{in} + W_\text{on}
\]

$Q_\text{in} = 0$ as there is no heat transfer, hence:
\[
    \Delta U = W_\text{on}
\]

\subsection{Derivations}
\begin{proof}
    For adiabatic transitions, where $Q = 0$.

    The First Law becomes $\Delta U = Q_\text{in} + W_\text{on}$. Since $Q_\text{in} = 0$:
    \[
        dU = 0 + dW_\text{on}
    \]

    And assuming work is being done \emph{on} the gas, the sign becomes negative:
    \[
        dU = -P dV
    \]
    \[
        \frac{dU dT}{dT} = -P dV
    \]

    We slightly abuse the definition of heat capacity here, and this will be elaborated on in a later lecture:
    \[
        C_V dT = -P dV
    \]
    
    Using $PV = nRT$, we have: $d(PV) = nR dT$:
    \[
        nR dT = P dV + V dP
    \]
    \[
        dT = \frac{P dV + V dP}{nR}
    \]
    \[
        C_V dT = - PdV
    \]
    \[
        0 = C_V \frac{P dV + V dP}{nR} + PdV
    \]
    \[
        0 = C_V \left(P dV + V dP\right) + nR pdV
    \]
    \[
        0 = P dV \left(C_v + nR\right) + C_V V dP
    \]
    \[
        0 = P dV (C_P) + V dP (C_V)  
    \]
    \[
        0 = \frac{C_P}{C_V} PdV + V dP
    \]

    Let $\gamma = \frac{C_P}{C_V}$, and dividing through by pressure and volume:
    \[
        \gamma \frac{dV}{V} + \frac{dP}{Ps}
    \]
    \[
        \int \gamma \frac{dV}{V} = - \int \frac{dP}{P}
    \]
    \[
        \gamma \ln V + c_1 = - \ln(P) + c_2
    \]
    
    Finally:
    \[
        \gamma \ln(v) + \ln(p) = \text{const}
    \]
    \[
        \ln(P V^\gamma) = \text{const}
    \]
    \[
        PV^\gamma = e^\text{const} = \text{another constant}
    \]

    Hence:
    \[
        PV^\gamma = \text{const}
    \]
    
    As required!
\end{proof}