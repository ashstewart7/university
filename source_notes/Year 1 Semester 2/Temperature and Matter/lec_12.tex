% !TeX root = main.tex
\lecture{12}{Thu 26 Feb 2026 13:00}{Heat Transfer}
\section{Thermal Transport}
\begin{tcolorbox}
    \textbf{Newton's Law of Cooling:} ``The rate of heat loss of a body is directly proportional to the difference in the temperatures between the body and the environment''
\end{tcolorbox}

There are three means of thermal transport:
\begin{itemize}
    \item \textbf{Conduction:} Heat is transmitted directly from one material to an adjoining material (or across parts of the same material) if there is a temperature difference between the two. There is no movement of the material.
    \item \textbf{Convection: } Movement of particles through a substances (often particles that comprise a fluid) that transport their heat energy along with them.
    \begin{itemize}
        \item The Archimedes principle describes buoyancy based on the material's density $\rho$, $g$ and the volume of the the fluid displaced by the object, $V$:
        \[
            F_b = \rho g V
        \]

        \item In convection, heating a particle causes an increase in bond size (by the LJP), hence the density decreases. This causes it to displace gas of a larger density, hence it rises.
    \end{itemize}
    \item \textbf{Radiation: }  Emission of electromagnetic radiation by all bodies which have heat (i.e. all bodies full stop), depending on their temperature.
    
\end{itemize}

\subsection{Heat Baths}
A heat bath is an object with a sufficiently large heat capacity, $C$ such that a large change in its heat energy will result in a negligible change in temperature.

For example, a hot object placed in a very large bath of water (i.e. a whole swimming pool), or a building placed on a planet.

\section{Conduction}
Suppose we have a small uniform rod of length $L$ and cross-sectional area $A$ linking two heat baths. The rod is conducting and the heat baths have temperature $T_1, T_2$.

\begin{figure}[H]
    \centering
    \includegraphics[width=0.75\textwidth]{figures/lec12-01.png}
     \caption{}
\end{figure}

Eventually, we reach a ``steady state'' where all points on the road have a $Q$ which does not change with time.

\[
    \frac{dQ}{dt} = \dot{Q} = \text{constant}, \qquad \forall \text{x along rod (in steady state)}
\]

What does this value of $\dot{Q}$ depend on? For a small section $dx$ of the rod:
\begin{itemize}
    \item Temperature change of the small section, $dT$.
    \item Cross-sectional area of the rod $A$.
    \item The ``thermal conductivity'' of the rod, $\kappa$\footnote{We also have another very similar quantity labelled $K$ which will be covered later and may be a source of confusion}. It has units $\si{W \si{m^{-1} K^{-1}}}$.
\end{itemize}

Putting these together gives us Fourier's Law:
\[
    \dot{Q} = - \kappa A \frac{dT}{dX}
\]

As $T_2$ is hotter, the flow of heat is from right to left. Therefore until thermalisation occurs, temperature increases from left to right across the rod ($\frac{dT}{dx} > 0$). However, the heat flow is in the opposite direction from right to left, hence the negative sign in the formula for $\dot{Q}$.

We want to show that $\frac{dT}{dx} = \frac{T_2 - T_1}{L}$:
\begin{align*}
    \dot{Q} &= - \kappa A \frac{dT}{dX} \\
    \frac{dT}{dx} &= - \frac{\dot{Q}}{\kappa A} \\
    -\int \frac{\dot{Q}}{\kappa A} dx &= \int dT \\
    -\int_0^L \frac{\dot{Q}}{\kappa A} &= \int_{T_1}^{T_2} dT \\
    - \frac{\dot{Q}}{\kappa A} \left[x\right]_0^L &= T_2 - T_1 \\
    \frac{dT}{dx} &= \frac{T_2 - T_1}{L}
\end{align*}

\subsection{Thermal Conductivity}
A higher value of $\kappa$ means a material is more thermally conductive, for example:
\begin{itemize}
    \item Diamond (natural): $\kappa = 2200 \si{W m^{-1} K^{-1}}$.
    \item Copper: $\kappa = 400 \si{W m^{-1} K^{-1}}$.
    \item Air: $\kappa = 0.02 \si{W m^{-1} K^{-1}}$.
    \item Aerogel: $\kappa = 0.003 \si{W m^{-1} K^{-1}}$.
\end{itemize}

Diamond is highly thermally conductive due to high frequency vibrations (``phonons'') being able to propagate through the material as a result of its rigidity.

There are two sources of heat conductivity, one being as a result of electrons and one as a result of phonons.
\begin{itemize}
    \item Inside the material, we have two bands - valence and conduction. These bands are a thick collection of many energy levels very close together. As the Pauli exclusion principle says that no two electrons can share a quantum number, each level must shift infinitesimally to not overlap. The levels within each band are so close that we treat electron as being able to move freely between them.
    \item The valence band is where electrons are still bound to an atom, the conduction band is where they are bound to the material as a whole and are free to move through the material.
    \item For an insulator, the difference between the two bands is $\sim 10 \si{eV}$. This represents a very small proportion of particles with sufficient energy and results in very little thermal excitation (hence little thermal conduction). As an electron is excited, it leaves a hole behind (that we treat as a particle $h^+$ for ease). We treat this hole as being able to move (as electrons themselves move).
    \item For an electron to fall back to the valence band, it must fall into a hole. In an insulator the conduction band is effectively empty. For a semiconductor, this gap is more like $\sim 1$eV, so a much higher number of electrons can jump, but this is still temperature dependant.
    \item In metals, the conduction and valence bands have little to no gap. Therefore, all electrons are free conduction electrons.
\end{itemize}

Diamonds sit on the boundary between being an insulator and a semiconductor. They work by phonons. As we heat a material, bonds in the material vibrate. As diamond is very rigid, vibrating one bond causes a ripple effect - effectively an wave of excitation (where one of these waves is called a phonon). This phonon propagation causes the transfer of vibrations hence the transfer of heat energy.

\subsection{Thermal Properties}
We can also define the:
\begin{itemize}
    \item Thermal Resistivity: $\rho = 1/\kappa$.
    \item Thermal Conductance $K = \kappa A / L$. \footnote{This is the aforementioned annoying symbol overlap between $\kappa$ and K}
    \item Thermal Resistance: $R = L / (\kappa A)$.
\end{itemize}

This are the heat analogues of the equivalent terms for electrical conductance in a circuit.

Annoyingly, thermal conductivity also varies with temperature\dots It also does so in a really weird way depending on the material:
\begin{figure}[H]
    \centering
    \includegraphics[width=0.75\textwidth]{figures/lec12-02.png}
     \caption{$\kappa$ for various materials, as a function of temperature.}
\end{figure}

This comes from phonon excitation and electron excitation being two different competing process that both have different degrees of contribution at different temperatures.

This means that Newton's Law of Cooling no longer holds for $\kappa = f(T)$, as the proportionality no longer holds\dots
