% !TeX root = main.tex
\lecture{11}{Tue 24 Feb 2026 12:00}{PV Planes and Transitions}

\textbf{Note RE: Last Lecture}\\
To avoid confusion, the Physics convention is:
\[
    W_\text{by} = \int P dV
\]
\[
    W_\text{on} = - \int P dV
\]

\section{PV Diagrams}
We can move between points on a PV plane through processes which usually involve some kind of heat transfer and work. For example:
\begin{figure}[H]
    \centering
    \includegraphics[width=0.75\textwidth]{figures/lec11-01.png}
     \caption{}
\end{figure}

The two transitions from $A \to B$ are processes, and the nature of the process depends on the shape of the transition on the PV plot.

\begin{itemize}
    \item Since the two possible transitions between A and B share the same start and end point, the change in internal energy $U_{AB}$ is the same for both. Since we go from $T_A$ to $T_B$ in both cases (as $U$ depends on $T$, which in turn depends on $P, V$ so for the same $P, V$ we have the same $T$), both paths have the same start/end internal energy, so the same change between them.
    \item The work done by the gas in the two paths is not the same, however. $W_\text{by}$ is larger for the blue path as $W$ is an integral of $\int P dV$, which represents the area under the curve. The blue line is further up the pressure axis, so the integral is larger and more work is done.
    \item What about heat input to the gas, $Q_\text{in}$? Since the First Law says $\Delta U = Q_\text{in} + W_\text{on} \implies \Delta U = Q_\text{in} - W_\text{by}$.
    
    $\Delta U$ is constant, and $-W_\text{by}$ has a larger magnitude for blue, so must $Q_\text{in}$ to keep $\Delta U$ constant.
\end{itemize}

These transitions are called a \emph{PV Cycle} as we can start at $A$, go to $B$ and end back up at $A$.

Consider a simple example with straight lines:
\begin{figure}[H]
    \centering
    \includegraphics[width=0.75\textwidth]{figures/lec11-02.png}
     \caption{}
\end{figure}

Where $1 \to 2$ and $3 \to 4$ are isochoric processes and $2 \to 3$ and $4 \to 1$ are isobaric. We can also look at isothermal and adiabatic processes:
\begin{figure}[H]
    \centering
    \includegraphics[width=0.75\textwidth]{figures/lec11-03.png}
     \caption{}
\end{figure}

A cycle must follow these rules:
\begin{enumerate}
    \item $\Delta U = 0$, for a full cycle.
    \item Total work done is given by the area contained by the cycle.
    \item Clockwise cycle for positive work, anticlockwise for negative work.
    \item $Q_\text{in} + W_\text{on} = 0$ as $\Delta U = 0$.
\end{enumerate}

\section{Otto Cycle}
This describes a petrol internal combustion engine:
\begin{figure}[H]
    \centering
    \includegraphics[width=0.75\textwidth]{figures/lec11-04.png}
     \caption{Blue: Isobaric, Green: Isochoric, Red: Adiabatic}
\end{figure}

This has the following components:
\begin{enumerate}
    \item $1 \to 2$: Intake stroke: Isobaric introduction of air and fuel to the system.
    \item $2 \to 3$: Adiabatic compression of the fluid as the piston compresses it.
    \item $3 \to 4$: Isochoric introduction of heat energy to the system (ignition of the fluid mixture).
    \item $4 \to 5$: Adiabatic expansion of the fluid mixture as the gas does work on the piston (pushing it down).
    \item $5 \to 2$: Isochoric expulsion of heat from the system.
    \item $2 \to 1$: Mass of air and exhaust gases released at a constant pressure.
\end{enumerate}

\section{Diesel Cycle}
Here, the ignition and combustion process is performed at a constant pressure where work is done by the gas:
\begin{figure}[H]
    \centering
    \includegraphics[width=0.75\textwidth]{figures/lec11-05.png}
     \caption{Blue: Isobaric, Green: Isochoric, Red: Adiabatic}
\end{figure}

In both cases, heat is introduced (by combustion) from $3 \to 4$, denoted $Q_1$ and heat is lost from $5 \to 2$ when heat is expelled from the system. The useful work done is $Q_1 - Q_2$ and is given by the area enclosed by the loop.

\subsection{Example}
Consider the adiabatic gas compression stroke in a diesel engine. (This is the adiabatic stroke where pressure is increasing/volume is decreasing, i.e. from $2 \to 3$.)

$10^{-3} \si{m^3}$ of nitrogen gas is initially at atmospheric pressure and $298 \si{K}$ and is compressed to $1/15$th of its original volume.

You may assume that nitrogen is an ideal diatomic gas with $\gamma=1.4$ and $C_V = 20.85 J/Kmol$.

Calculate:
\begin{enumerate}
    \item The final pressure.
    \item The final temperature.
    \item The number of moles in the gas.
    \item The change in internal energy.
    \item The work done by the gas.
\end{enumerate}

\textbf{Part I}\\
Atmospheric pressure is $101325 \approx 10^5\si{Pa}$. As this is an adiabatic transition, $PV^\gamma = \text{const}$. Hence:
\[
    P_1 V_1^\gamma = P_2 V_2^\gamma
\]
\[
    \implies P_2 = P_1 \left(\frac{V_1}{V_2}\right)^\gamma
\]
\[
    \implies P_2 = P_1 \left(\frac{V_1}{(V_1 / 15)}\right)^\gamma
\]
\[
    \implies P_2 = P_1 \left(15\right)^\gamma = 4.5 \si{\mega\pascal}
\]

\textbf{Part II}\\
We can use either $PV = nRT$ twice, using the initial conditions to determine $n$ or we can use:
\[
    T_1 V_1^{(\gamma - 1)} = T_2 V_2^{(\gamma - 1)}
\]

Using the same process gives:
\[
    T_2 = T_1 (15)^{(\gamma - 1)} = 880 \si{K}
\]

\textbf{Part III}\\
Since we know $P_1, V_1$ or $P_2, V_2$ we can use $PV = nRT$, as mentioned above. It would be a good sanity check to do both and compare the values. 

\textbf{Part IV}\\
It's an adiabatic transition, we can related $C_V = \frac{\partial U}{\partial T}$.

Integrating both sides gives:
\[
    \int dT = \int \frac{\partial U}{\partial T}
\]

