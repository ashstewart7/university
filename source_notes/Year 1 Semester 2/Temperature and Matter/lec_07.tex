% !TeX root = main.tex
\lecture{7}{Mon 09 Feb 2026 11:00}{Introduction to Temperature}
% TODO, THE END OF LECTURE 8 STUFF

We're starting the portion of the module on thermodynamics, looking at temperature, pressure etc and transitions involving these.
\section{What is Temperature?}
Particle for a single particle isn't well defined for the definition of temperature we want to use. There is two definitions:
\begin{itemize}
    \item One is statistical, looking at average kinetic energy of particles.
    \item One is based on entropy and isn't looked at until second year.
\end{itemize}

We define temperature as \emph{a statistical collection of energies for particles that make up the system for which we are measuring the temperature}. A material will have a range of kinetic energies, some very large, some very small, so we care about a statistical average. Thermalisation happens as a result of these high kinetic energy particles striking lower kinetic energy particles.

\begin{tcolorbox}
    \textbf{Temperature: } A statistical collection of energies of the particles that comprise the system in question.
\end{tcolorbox}

We can express temperature as using the Maxwell-Boltzmann distribution:
\[
    P(v) = 4 \pi \left(\frac{M}{2 \pi RT}\right)^\frac{3}{2} v^2 \exp \left(\frac{- Mv^2}{2RT}\right)
\]

We look at this in more depth when we cover statistical physics later, but for now we care about the key features. If we plot $P(v)$ (the probability of finding a particle at certain velocity) against these velocities for a range of temperatures:
\begin{figure}[H]
    \centering
    \includegraphics[width=0.75\textwidth]{figures/lec07-1.png}
     \caption{}
\end{figure}

This has exponential decay, so theoretically we can find a particle at any velocity within a gas of any temperature. Particles with higher temperatures have a higher peak, so a higher average velocity.

\section{Temperature Scales}
We can use a number of different temperature scales:
\begin{itemize}
    \item The Fahrenheit scale is set such that the freezing point of water is $98.6^\circ$F, and the boiling point of water at $212^\circ$F. It's designed to be based around the range of temperatures that a person could realistically could experience in the core range from 0-200. It's not a useful unit and won't be used in exams.
    \item Celsius is the standard unit.
    \item Kelvin is the common unit of thermodynamics. It has the same graduations as degrees celsius but is shifted such that absolute zero is at $0$K. Note that Kelvin has no degrees symbol. $\si{\degreeCelsius} \to \si{\kelvin}$ can be converted between using:
    \[T (\si{\kelvin}) = T (\si{\degreeCelsius}) + 273.15\]
\end{itemize}

\section{Thermal Expansion}
As a solid is heated, it expands in all directions. Consider a cuboid of height, width, depth $W, L, D$, we have new dimensions of $D + \Delta W, \, L + \Delta L, \, D + \Delta D$ after a small change in temperature. We can define the linear expansion coefficient (a material property) $\alpha_L$ based on these dimensions and the change in temperature:
\[
    \alpha_L \Delta T = \frac{\Delta L}{L} = \frac{\Delta W}{W} = \frac{\Delta D}{D}
\]

If we consider the area instead, we have the area expansion coefficient:
\[
    \frac{\Delta A}{A} = \alpha_A \Delta T \approx 2 \alpha_L \Delta T
\]

The latter approximation can be derived as follows. Consider one dimension, i.e. $L$:
\[
    \frac{\Delta}{L} = \alpha_L \Delta T
\]
\[
    \implies \Delta L = L \alpha_L \Delta T
\]

So:
\[
    L_\text{new} = L + \Delta L = L \left(1 + \alpha_L \Delta T\right)
\]

Pairing this with width to find an area (of a face):
\[
    W_\text{new} = W + \Delta W = W \left(1 + \alpha_L \Delta T\right)
\]
\[
    A_\text{old} = L \times W
\]
\[
    A_\text{new} = L_n \times W_n
\]
\[
    A_n = LW \left(1 + \alpha_L \Delta T\right)^2
\]
\[
    = LW \left(1 + 2 \alpha_L \Delta T + \alpha_L^2 \Delta T^2\right)
\]

For a small change in temperature, the second order term is very small, hence:
\[
    \frac{\Delta(LW)}{LW} \approx 2 \alpha_L \Delta T
\]

The linear expansion coefficient has units of $\si{\kelvin^{-1}}$.

\subsection{Thermal Stress}
Consider a rod fixed between two immovable walls. The rod attempts to expand, but is fixed. This provides a thermal stress on the wall. If the rod has cross-sectional area $A$ and Young Modulus $Y$, the thermal stress is given by:
\[
    \frac{F}{A} = Y \alpha_L \Delta T
\]

This is why bridges etc need to have thermal expansion joints, otherwise they would buckle/break under this thermal stress.

\subsection{Lennard Jones and Thermal Expansion}
% TODO AFTER I'VE COME BACK TO LENNARD JONES