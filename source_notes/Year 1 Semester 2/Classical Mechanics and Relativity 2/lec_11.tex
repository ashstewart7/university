% !TeX root = main.tex
\lecture{11}{Tue 24 Feb 2026 11:00}{Special Relativity III: Lorentz Transformations}

Recap of the first two relativity lectures:
\begin{itemize}
    \item Recap from Semester 1 Special Rel:
    \begin{itemize}
        \item Background
        \item Einstein's postulates.
        \item Time dilation.
        \item Lorentz contraction.
    \end{itemize}
    \item Galilean transformations and when they fail.
    \item Lorentz transformations for space and time.
\end{itemize}

\section{Galilean vs. Lorentz Transformations}
In a standard setup with two frames, $\Sigma$ and $\Sigma^\prime$, where $\Sigma^\prime$ moves with speed $v$ relative to $\Sigma$.

In $\Sigma$, we have coordinates $(x, y, t)$, in $\Sigma^\prime$ we have coordinates $(x^\prime, y^\prime, ^\prime)$.

In a Galilean transformations, we kept time invariant across both frames, so $t = t^\prime$. This lead to $x^\prime = x - vt \implies u^\prime = u - v$ but raised a contradiction with the speed of light as observers in different frames would take two measurements of the speed of light.

In the second lecture, we consider the speed of light as being invariant across all frames instead of time to resolve the contradiction. This gave us the Lorentz Transformations:
\[
    x^\prime = \gamma(x - vt)
\]
\[
    t^\prime = \gamma \left( t = \frac{xv}{c^2}\right)
\]

Where $y = y^\prime$, $z=^\prime$.

As we cannot use $u^\prime = u - v$ due to the contradiction with the speed of light, we in this lecture will consider:
\begin{itemize}
    \item Lorentz transformations for velocity.
    \item Space-time, Lorentz invariance, causality.
\end{itemize}

\section{Lorentz Transformation for Velocity}
We can revisit the question from the first lecture, but in the Lorentz transformation perspective rather than a Galilean perspective.

Again, we have our two standard frames $\Sigma, \Sigma^\prime$. An object is moving with speed $u^\prime$ in frame $\Sigma^\prime$. We want to find an expression for the velocity of the object as viewed from $\Sigma$, given by $u$. The prime frame moves at speed $v$ wrt the rest frame and the objects motion is entirely along the $x$ acis.

By definition, we have:
\[
    u = \frac{dx}{dt} = \frac{dx}{dt^\prime} \frac{dt^\prime}{dt} = \frac{\left(\frac{dx}{dt^\prime}\right)}{\left(\frac{dt^\prime}{dt}\right)} \tag{1}
\]

And:
\[
    \frac{dx}{dt^\prime} = \gamma \left(\frac{dx^\prime}{dt^\prime} + v\right) = \gamma(y^\prime + v) \tag{2}
\]
\[
    \frac{dt}{dt^\prime} = \gamma \left(1 + \frac{v}{c^2} \frac{dx^\prime}{dt^\prime}\right) = \gamma \left(1 + \frac{u^\prime v}{c^2}\right) \tag{3}
\]

And substituting (3) and (2) into (1):
\[
    u = \frac{\cancel{\gamma} (u^\prime + v)}{\cancel{\gamma} \left(1 + \frac{u^\prime v}{c^2}\right)} \implies \boxed{u = \frac{u^\prime + v}{1 + \frac{u^\prime v}{c^2}}}
\]

To get the reverse transformation, we invoke forward-backward symmetry, i.e. swap $x \to x^\prime$, $t\to t^\prime$ etc, where $v \to -v^\prime$:
\[
    u^\prime = \frac{u-v}{1 - \frac{uv}{c^2}}
\]

\subsection{Limiting Cases}
Firstly, we'll look at the non-relativistic case, i.e. where $u, v \ll c$:
\[
    u = \frac{u^\prime+v}{1 + \frac{u^\prime v}{c^2}} = \frac{u^\prime + v}{1 + \text{small}} \approx u^\prime + v
\]

This is the classical Galilean approach, which is what we'd expect for non-relativistic behaviour.

What about an ultra-relativistic limit where $u, u^\prime \to c$ with $v \ll c$. Now we have:
\[
    u = \frac{u^\prime + \text{small}}{1 + \text{small}} \approx u^\prime
\]

As the speed of light is invariant, speeds close to the speed of light do not change between reference frames.

\subsection{Example}
An observer on a rocket sees a second rocket moving away from them at $v_1 = 0.8c$. Another observer on a planet sees the first rocket moving at $0.7c$. Assuming all motion is in the same direction, how fast does the planetary observer see the second rocket?

\begin{itemize}
    \item From $\Sigma^\prime$ (the first rockets rest frame), the second rocket moves with $0.8c$.
    \item From $\Sigma$ (the planetary rest frame), the first rocket moves with $0.7c$, and the second rocket moves with some unknown speed $u$.
\end{itemize}

We apply the formula:
\[
    u = \frac{u^\prime + v}{1 + \frac{u^\prime v}{c^2}}
\]

In $\Sigma^\prime$, $u^\prime = 0.8c$. The two frames move with relative velocity equal to the change in speed of the first rocket from $\Sigma \to \Sigma^\prime$: $v = 0.7c$.

In frame $\Sigma$ we are tying to do:
\[
    u = \frac{0.8c + 0.7c}{1 + \frac{0.8c \times 0.7c}{c^2}}
\]
\[
    = \frac{1.5c}{1.56} = 0.96c
\]

\section{Space-Time and Lorentz Invariants}
From length contraction and time dilation, we can see that in special relativity, intervals in time and space are not fixed but vary from frame-to-frame.

This means that the order of events may even be different for different observers. 

There are some quantities however which are invariant across frames, such as the speed of light. One of these is space-time.

\subsection{Space-Time Invariance}
As usual, we have $\Sigma$ and $\Sigma^\prime$ with relative speed $v$. The axes of these frames coincide at $t = 0, x =0$ and $t^\prime = 0, x^\prime = 0$.

At $t=t^\prime=0$, there is a flash of light at the origin which propagates isotropically.

% TOOD DIAGRAM

In $\Sigma$, the light forms a spherical shell of radius $\Delta r$ at time $\Delta t$, and in $\Sigma^\prime$ it reaches a radial distance $\Delta r^\prime$ at time $\Delta t^\prime$. 

After time $\Delta T$ in $\Sigma$ we have $\Delta r = c \delta T$. Similarly in $\Sigma^\prime$ we have $\Delta r^\prime = c \Delta t^\prime$.

Squaring these we have:
\[
    \left(\Delta r\right)^2 = c^2 \Delta t^2 = \Delta x^2 + \Delta y^2 + \Delta z^2
\]
\[
    \left(\Delta r^\prime\right)^2 = c^2 \Delta t^{\prime 2} = \Delta x^{\prime 2} + \Delta y^{\prime 2} + \Delta z^{\prime 2}
\]

Hence:
\[
    c^2 \Delta t - \Delta x^2 - \Delta y^2 - \Delta z^2 = 0 = c^2 \Delta t^{\prime 2} - \Delta x^{\prime 2} - \Delta y^{\prime 2} - \Delta z^{\prime 2}
\]

This means that, \textbf{for light}:
\[
    c^2 \Delta t^2 - \Delta x^2 - \Delta y^2 - \Delta z^2 = 0 
\]

We call this quantity $\Delta s^2$:

What about for objects moving at a speed less than the speed of light? We want to apply the Lorentz transformations to $\Delta s{^\prime 2}$ (again treating the ``boost'' as being in $x$):

\begin{align*}
    \Delta s^{\prime 2} &= c^{\prime 2} \Delta t^{\prime 2} - \Delta x^{\prime 2} - \Delta y^{\prime 2} - \Delta z^{\prime 2} \\
    &= c^2 \gamma^2 \left(\Delta t - \frac{v}{c^2} \Delta x\right)^2 - \gamma^2 \left(\Delta x - v \Delta t\right)^2 - \Delta y^2 - \Delta z^2\\
    &= c^2 \gamma^2 \left(\Delta t^2 + \frac{v^2}{c^4} \Delta x^2 - 2 \frac{v}{c^2} \Delta t \Delta x\right) - \gamma^2 \left(\Delta x^2 + v^2 \Delta t^2 - 2v \Delta x \Delta t\right) - \Delta y^2 - \Delta z^2 \\
    &= \Delta t^2 \left(c^2 \gamma^2 - v^2 \gamma^2\right) + \Delta x^2 \left(\frac{v^2 \gamma^2}{c^2}- \gamma^2\right) + \Delta t \Delta x \left(-2v \gamma^2 + 2v \gamma^2\right) - \Delta y^2 - \Delta z^2\\
    &= \Delta t^2 \gamma^2 \left(c^2 - v^2\right) + \Delta x^2 \gamma^2 \left(\frac{v^2}{c^2} - 1\right) - \Delta y^2 - \Delta z^2 \\
    &= \Delta t^2 \frac{c^2 - v^2}{1 - \frac{v^2}{c^2}} + \Delta x^2 \frac{(\frac{v^2}{c^2}) - 1}{1 - \frac{v^2}{c^2}} - \Delta y^2 - \Delta z^2 \\
    \Delta s^{\prime 2} &= \boxed{c^2 \Delta t^2 - \Delta x^2 - \Delta y^2 - \Delta z^2 = \Delta s^2}
\end{align*}


Hence $\Delta s^2$ is invariant under the Lorentz transformations regardless of speed.















